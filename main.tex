%%%%%%%%%%%%
%% Please rename this main.tex file and the output PDF to
%% [lastname_firstname_graduationyear]
%% before submission.
%%
%% This .tex file is for use with BibLaTeX. Please use
%% main-bibtex.tex instead if you prefer BibTeX.
%%%%%%%%%%%%

\documentclass[12pt]{caltech_thesis}
\usepackage[hyphens]{url}
\usepackage{hyperref}
\usepackage{lipsum}
\usepackage{graphicx}

\usepackage{todonotes}

%% Tentative: newtx for better-looking Times
\usepackage[spanish]{babel}
\usepackage[utf8]{inputenc}
\usepackage[T1]{fontenc}
\usepackage{newtxtext,newtxmath}

%% Added by penserbjorne
\usepackage{csquotes}

% Must use biblatex to produce the Published Contents and Contributions, per-chapter bibliography (if desired), etc.
\usepackage[
    backend=biber,natbib,
    % IMPORTANT: load a style suitable for your discipline
    style=authoryear
]{biblatex}

% Name of your .bib file(s)
\addbibresource{citas.bib}
%\addbibresource{ownpubs.bib}

\begin{document}

%%%%%%%%%%%%%%%%%%%%%%%%%%%%%%%%%%%%%%%%%%%%%%%%%%%%%%%%%%%%%%%%%%%%%%%%%%%%%%%%
%   Portada
%%%%%%%%%%%%%%%%%%%%%%%%%%%%%%%%%%%%%%%%%%%%%%%%%%%%%%%%%%%%%%%%%%%%%%%%%%%%%%%%

% Do remember to remove the square bracket!
\title{Informe de trabajo profesional sobre el \\diseño e implementación de un laboratorio de amenazas digitales para sociedad civil con software libre y de código abierto}
\author{Paul Sebastian Aguilar Enriquez}

\degreeaward{Ingeniero en Computación}                 % Degree to be awarded
\university{Universidad Nacional Autónoma de México}    % Institution name
\address{Ciudad de México}                     % Institution address
\unilogo{img/01-logo-unam.png}                                 % Institution logo
\copyyear{2024}  % Year (of graduation) on diploma
\defenddate{[Exact Date]}          % Date of defense

%%  If you'd like to remove the Caltech logo from your title page, simply remove the "[logo]" text from the maketitle command
\maketitle[logo]
%\maketitle

%%%%%%%%%%%%%%%%%%%%%%%%%%%%%%%%%%%%%%%%%%%%%%%%%%%%%%%%%%%%%%%%%%%%%%%%%%%%%%%%
%   Licenciamiento
%%%%%%%%%%%%%%%%%%%%%%%%%%%%%%%%%%%%%%%%%%%%%%%%%%%%%%%%%%%%%%%%%%%%%%%%%%%%%%%%

%\orcid{[Author ORCID]}

%% IMPORTANT: Select ONE of the rights statement below.
%\rightsstatement{All rights reserved}
%\rightsstatement{All rights reserved except where otherwise noted}
\rightsstatement{Esta obra está bajo una licencia \\ ``Creative Commons Atribución-NoComercial-CompartirIgual 4.0 Internacional``.}



%%%%%%%%%%%%%%%%%%%%%%%%%%%%%%%%%%%%%%%%%%%%%%%%%%%%%%%%%%%%%%%%%%%%%%%%%%%%%%%%
%   Agradecimientos
%%%%%%%%%%%%%%%%%%%%%%%%%%%%%%%%%%%%%%%%%%%%%%%%%%%%%%%%%%%%%%%%%%%%%%%%%%%%%%%%

\begin{acknowledgements} 	 
   [Add acknowledgements here. If you do not wish to add any to your thesis, you may simply add a blank titled Acknowledgements page.]
\end{acknowledgements}

%%%%%%%%%%%%%%%%%%%%%%%%%%%%%%%%%%%%%%%%%%%%%%%%%%%%%%%%%%%%%%%%%%%%%%%%%%%%%%%%
%   Resumen
%%%%%%%%%%%%%%%%%%%%%%%%%%%%%%%%%%%%%%%%%%%%%%%%%%%%%%%%%%%%%%%%%%%%%%%%%%%%%%%%

\begin{abstract}
   [This abstract must provide a succinct and informative condensation of your work. Candidates are welcome to prepare a lengthier abstract for inclusion in the dissertation, and provide a shorter one in the CaltechTHESIS record.]
\end{abstract}

%%%%%%%%%%%%%%%%%%%%%%%%%%%%%%%%%%%%%%%%%%%%%%%%%%%%%%%%%%%%%%%%%%%%%%%%%%%%%%%%
%   Contenido y contribuciones publicadas
%%%%%%%%%%%%%%%%%%%%%%%%%%%%%%%%%%%%%%%%%%%%%%%%%%%%%%%%%%%%%%%%%%%%%%%%%%%%%%%%

%% Uncomment the `iknowhattodo' option to dismiss the instruction in the PDF.
%\begin{publishedcontent}%[iknowwhattodo]
% List your publications and contributions here.
%\nocite{Cahn:etal:2015,Cahn:etal:2016}
%\end{publishedcontent}

%%%%%%%%%%%%%%%%%%%%%%%%%%%%%%%%%%%%%%%%%%%%%%%%%%%%%%%%%%%%%%%%%%%%%%%%%%%%%%%%
%   Índices
%%%%%%%%%%%%%%%%%%%%%%%%%%%%%%%%%%%%%%%%%%%%%%%%%%%%%%%%%%%%%%%%%%%%%%%%%%%%%%%%

\tableofcontents
\listoffigures
\listoftables
\printnomenclature

\mainmatter

%%%%%%%%%%%%%%%%%%%%%%%%%%%%%%%%%%%%%%%%%%%%%%%%%%%%%%%%%%%%%%%%%%%%%%%%%%%%%%%%
%   http://escolar.ingenieria.unam.mx/_adicionales/titulacion/GuiaInformeTrabajo2019.pdf
%
%   El informe de trabajo profesional es un documento que describe las actividades
%   profesionales que durante o al término de sus estudios desarrolla un alumno por
%   al menos un semestre. Deberá redactarse en primera persona del singular y en
%   pasado, y contemplar algunos de los puntos que se mencionan a continuación,
%   con los que a juicio del asesor y del alumno se demuestren sus capacidades y
%   competencias para el ejercicio de la profesión.
%
%   Por temas legales y/o de confidencialidad laboral y/o de propiedad intelectual,
%   en el trabajo escrito y anexos se debe omitir información específica de la
%   empresa, proyectos, productos, marcas comerciales, etc., procurando emplear
%   en su lugar términos generales al referirse a estos tópicos (“una empresa”, “un
%   proyecto”, “el diseño de un producto”, etc.).
%%%%%%%%%%%%%%%%%%%%%%%%%%%%%%%%%%%%%%%%%%%%%%%%%%%%%%%%%%%%%%%%%%%%%%%%%%%%%%%%

%%%%%%%%%%%%%%%%%%%%%%%%%%%%%%%%%%%%%%%%%%%%%%%%%%%%%%%%%%%%%%%%%%%%%%%%%%%%%%%%
%   Introducción y Objetivo
%   Presentar de forma clara, breve y precisa el
%   contenido del informe, que puede corresponder a un proyecto de mejora,
%   propuesta de cambio o actividades profesionales diversas desempeñadas
%   en el puesto que ocupó, y definir el objetivo de la propuesta, mostrando
%   la capacidad para aplicar los conocimientos adquiridos en su formación
%   como ingeniero.
%%%%%%%%%%%%%%%%%%%%%%%%%%%%%%%%%%%%%%%%%%%%%%%%%%%%%%%%%%%%%%%%%%%%%%%%%%%%%%%%

\chapter{Introducción y Objetivo}

%[Presentar de forma clara, breve y precisa el contenido del informe, que puede corresponder a un proyecto de mejora, propuesta de cambio o actividades profesionales diversas desempeñadas en el puesto que ocupó, y definir el objetivo de la propuesta, mostrando la capacidad para aplicar los conocimientos adquiridos en su formación como ingeniero.]

Este informe detalla mi experiencia como coordinador del área de Seguridad Digital y Tecnología en una ONG\nomenclature[ONG]{Organización No Gubernamental}{Pendiente}, enfocada en empoderar a agentes de cambio social como periodistas, activistas y personas defensoras de derechos mediante la tecnología.

Mi principal contribución ha sido el desarrollo de un Laboratorio de Amenazas\nomenclature[Laboratorio de Amenazas]{Pendiente}{Pendiente} usando software libre y de código abierto (FOSS)\nomenclature[FOSS]{Free and Open Source Software}{Pendiente}, destinado a fortalecer la respuesta y documentación ante incidentes\nomenclature{Incidentes}{Pendiente} o ataques digitales\nomenclature{Ataques Digitales}{Pendiente} que afectan a los grupos e individuos con quienes trabajamos en México y América Latina.

A través de este proyecto, he aplicado habilidades clave adquiridas durante mi formación en Ingeniería en Computación, incluyendo análisis y solución de problemas, diseño de infraestructuras web, gestión de software y bases de datos, y protocolos de seguridad informática.

Este trabajo no solo refleja mi capacidad técnica, sino también mi compromiso con la aplicación práctica de la tecnología para el bien social, alineándome con los objetivos de mi carrera profesional.

%Here's an example of a citation \citep{GMP81}. Here's another \citep{PP98}. These will appear in the big bibliography at the end of the thesis.

%\begin{figure}[hbt!]
%    \centering
%    \includegraphics[width=.3\textwidth]{img/caltech.png}
%    \caption{This is a figure}\label{fig:logo}
%\end{figure}

%\begin{table}[hbt!]
%    \centering
%    \begin{tabular}{ll}
%        \hline
%        Area & Count\\
%        \hline
%        North & 100\\
%        South & 200\\
%        East & 80\\
%        West & 140\\
%        \hline
%    \end{tabular}
%    \caption{This is a table}\label{tab:sample}
%\end{table}

%Here's an endnote.\endnote{Endnotes are notes that you can use to explain text in a document.}

%%%%%%%%%%%%%%%%%%%%%%%%%%%%%%%%%%%%%%%%%%%%%%%%%%%%%%%%%%%%%%%%%%%%%%%%%%%%%%%%
%   Contexto
%   Capitulo añadido por mi para contextualizar el trabajo que se desarrolla
%%%%%%%%%%%%%%%%%%%%%%%%%%%%%%%%%%%%%%%%%%%%%%%%%%%%%%%%%%%%%%%%%%%%%%%%%%%%%%%%

\chapter{Contexto social}

\section{Sobre agresiones y ataques digitales en América Latina}

Durante mi experiencia laboral me dediqué a estudiar y contrarrestar las agresiones digitales dirigidas a periodistas, activistas y personas defensoras de derechos en América Latina, una región donde gobiernos, crimen organizado y ciertas corporaciones privadas destacan por ser los principales atacantes. Estas prácticas, que se extienden tanto en el espacio físico como en el digital, han sido bien documentadas \citep{AmnestyInternational-Mexico-2023, Articulo19-Reporte-2022, Articulo19-Reporte-2023, CIMAC-Informe-2022, Internews-Reporte-2023}.

En mi rol, observé y abordé diversas formas de ataques digitales, desde agresiones en redes sociales\nomenclature{Agresiones en redes sociales}{Pendiente.}, doxing\nomenclature{Doxing}{Pendiente.}, intentos de acceso no autorizados a cuentas\nomenclature{Acceso no autorizado}{Pendiente}, phishing\nomenclature{Phishing}{Pendiente.}, ataques de denegación de servicios \nomenclature{DoS}{Ataque de Denegación de servicios} a sitios web, intervención de lineas de comunicación\nomenclature{Intervención de lineas de comunicación}{Pendiente.}, o intervención de dispositivos\nomenclature{Intervención de dispositivos}{Pendiente.} a través de spyware\nomenclature{Spyware}{Pendiente.}, por mencionar algunos de los más habituales que se atienden y trabajan en la ONG. Mi enfoque principal fue el análisis y la mitigación de estos ataques, aplicando mis habilidades como ingeniero en computación y mi comprensión del contexto tecnológico y social de la región.

Observé que la evolución de los ataques digitales en América Latina ha creado un ecosistema complejo, con un entrelazado de componentes físicos, digitales y psicoemocionales, impactando tanto a quienes sufren directamente las agresiones como a quienes, como yo, se dedican a su atención y prevención \citep{ElPais-CeilioPineda-2021, FernandaGarrido-ViolenciaDigital-2020, NYT-JavierValdez-2018, UNESCO-SDPeriodismo-2016}.

Los registros de estas prácticas, usualmente llevados a cabo por la sociedad civil, periodistas y medios, muestran un alto nivel de sofisticación y una respuesta directa a los contextos locales específicos de cada país. Por ejemplo, se documentaron situaciones como el control de Internet en Venezuela \citep{VeSinFiltro-Reporte-2023} y el uso de monitoreo de fuentes abiertas\nomenclature{Fuentes abiertas}{Pendiente.} en países como Guatemala, Nicaragua y Bolivia para identificar a posibles disidentes \citep{CICIG-GuateNetCenters-2019, Articulo19-Guatemala-2021}. También se evidenció el uso de tecnología avanzada para espionaje en México, El Salvador y República Dominicana \citep{CitizenLab-Mexico-Reckless, CitizenLab-ElSalvador-2022, AmnestyInternational-RepublicaDominicana-2023}.

Mi experiencia en la detección y respuesta a estos ataques me ha proporcionado una comprensión profunda de la ciberseguridad y sus desafíos en el contexto latinoamericano en sociedad civil.

\section{Sobre atención ante agresiones y ataques digitales en América Latina}

En el contexto de sociedad civil\nomenclature{Sociedad Civil}{Pendiente.} en América Latina pude observar de primera mano la escasez de recursos técnicos avanzados en la región. Aunque existen algunas herramientas básicas y amigables para usuarios no técnicos, estos recursos solían ser insuficientes para abordar las necesidades complejas de los grupos vulnerables afectados por ataques y agresiones digitales.

En mi experiencia, los grupos locales de atención a estas agresiones frecuentemente carecen de recursos técnicos, económicos y tecnológicos adecuados, lo que repercutía en una asistencia limitada. Esto me llevó a desarrollar habilidades en la adaptación y optimización de recursos limitados para maximizar su impacto. Aunque muchos de los que brindaban asistencia en estos grupos no tenían formación técnica especializada, mi formación en ingeniería en computación me permitió aportar una perspectiva técnica crucial a sus esfuerzos.

Por otro lado, en mis interacciones con organizaciones especializadas, como grupos de investigación universitarios y empresas privadas, noté que, aunque poseían un alto nivel técnico, su enfoque era a menudo demasiado específico y distante de las realidades de los grupos locales. Esta brecha me motivó a trabajar en la construcción de puentes entre estos dos mundos, equilibrando la sofisticación técnica con la aplicabilidad práctica en contextos locales.

Las colaboraciones entre estos grupos eran desafiantes debido a sus diferentes visiones y prioridades. Sin embargo, mi rol en la facilitación de estas interacciones y en la adaptación de herramientas de software libre y código abierto (FOSS) para contextos locales me permitió desarrollar habilidades en la gestión de proyectos y en la comunicación interdisciplinaria. Mi enfoque siempre fue hacer estas herramientas accesibles para quienes carecían del conocimiento técnico especializado necesario para su uso, así como promover el intercambio de conocimientos y técnicas entre grupos especializados y locales para fortalecer la respuesta colectiva ante las agresiones digitales.

%%%%%%%%%%%%%%%%%%%%%%%%%%%%%%%%%%%%%%%%%%%%%%%%%%%%%%%%%%%%%%%%%%%%%%%%%%%%%%%%
%   Antecedentes
%   Enmarcar en el ámbito de la ingeniería los antecedentes
%   del problema, tema, actividades o proyecto en el que se trabajó.
%%%%%%%%%%%%%%%%%%%%%%%%%%%%%%%%%%%%%%%%%%%%%%%%%%%%%%%%%%%%%%%%%%%%%%%%%%%%%%%%

\chapter{Antecedentes}

En mi rol dentro del área de seguridad digital de la organización, me enfrenté a desafíos técnicos significativos en la atención a ataques digitales contra periodistas, activistas y personas defensoras de derechos. Estos ataques varían en complejidad y naturaleza, reflejando las capacidades técnicas del atacante y sus objetivos específicos.

Desde una perspectiva de ingeniería en computación, los ataques técnicos más comunes que abordamos incluyeron phishing, análisis de tráfico de red, recuperación de sitios web, análisis forense y detección de spyware. Estos análisis requirieron un enfoque interdisciplinario que combinara conocimientos de seguridad informática, análisis de datos, redes de computación, manejo de servidores, sistemas GNU/Linux y scripting\nomenclature{Scripting}{Pendiente.}.

Además de los ataques técnicos, enfrentamos ataques centrados en la interacción humana como extorsiones, acoso, amenazas, difusión de información, doxxing, o robo de cuentas. Estos también requerían una comprensión técnica, aunque se enfocaban más en el aspecto social de la ciberseguridad.

El tratamiento de los casos generalmente se basaba en la experiencia empírica de los analistas, combinando conocimientos técnicos con aprendizajes adquiridos en situaciones reales. Sin embargo, noté una falta de estructuración y metodología estándar en nuestros procesos, lo que me llevó a reconocer la necesidad de estandarizar el conocimiento técnico, los métodos de análisis y la documentación.

Aunado a lo anterior, se han realizado esfuerzos de capacitaciones y formaciones técnicas en la atención de ataques digitales a periodistas, activistas y personas defensoras, lo cual ha mejorado la situación descrita previamente, así como la participación del equipo en espacios técnicos relacionados al tema con otros grupos y organizaciones que realizan un trabajo similar.

%\begin{refsection}
%If you'd like to have separate bibliographies at the end of each chapter, put a \verb|refsection| around the material of each chapter, then cite as usual -- e.g.~\citep{GMP81,Ful83}. Then do a \verb|\printbibliography| just before the \verb|refsection| ends. \index{bibliography!by chapter}

%\printbibliography[heading=subbibliography]
%\end{refsection}

%%%%%%%%%%%%%%%%%%%%%%%%%%%%%%%%%%%%%%%%%%%%%%%%%%%%%%%%%%%%%%%%%%%%%%%%%%%%%%%%
%   Definición del problema o contexto de la participación profesional
%   Describir claramente el proyecto, propuesta de cambio, actividades o
%   problemática a resolver en el contexto de la ingeniería, así como sus
%   alcances.
%%%%%%%%%%%%%%%%%%%%%%%%%%%%%%%%%%%%%%%%%%%%%%%%%%%%%%%%%%%%%%%%%%%%%%%%%%%%%%%%

\chapter{Definición del problema o contexto de la participación profesional}

Durante mi estancia en la organización, identifiqué que el volumen y la complejidad de los casos atendidos mostraban un incremento significativo. Esta situación evidenciaba la ausencia de un protocolo estandarizado para la documentación y gestión interna de los casos, lo que generaba dificultades al profundizar o compartir información de inteligencia\nomenclature{Inteligencia de Amenazas}{Pendiente.} de amenazas entre miembros de la organización o con entidades aliadas.

Cabe destacar que en la región de América Latina, son escasos los grupos que disponen de metodologías consolidadas, desde una perspectiva técnica, para la gestión y documentación de ataques digitales en contextos de sociedad civil. Esta carencia se manifiesta en la limitada disponibilidad de documentación de referencia para la gestión de incidentes y en la escasez de registros que presenten datos de agresiones digitales de manera estandarizada.

Ante este panorama, propuse la creación de un laboratorio de amenazas digitales enfocado en la sociedad civil, con el objetivo de abordar agresiones digitales contra periodistas, activistas y personas defensoras de derechos. La propuesta del laboratorio se fundamentó en los siguientes criterios:

\begin{itemize}
    \item Desarrollar y adoptar documentación técnica para la gestión de incidentes, estableciéndola como referencia metodológica técnica.
    \item Implementar capacidades para la documentación y seguimiento de incidentes, enfocándose tanto en el contexto narrativo como en el técnico.
    \item Construir el laboratorio de tal manera que sirva tanto para uso interno de la organización como para la colaboración con otras entidades similares en la región.
    \item Utilizar herramientas basadas en software libre y de código abierto.
\end{itemize}

%\publishedas{Cahn:etal:2015}

%[You can have chapters that were published as part of your thesis. The text style of the body should be single column, as it was submitted to the publisher, not formatted as the publisher did.]

%%%%%%%%%%%%%%%%%%%%%%%%%%%%%%%%%%%%%%%%%%%%%%%%%%%%%%%%%%%%%%%%%%%%%%%%%%%%%%%%
%   Metodología utilizada
%   Describir los métodos, técnicas o procedimientos de ingeniería empleados.
%%%%%%%%%%%%%%%%%%%%%%%%%%%%%%%%%%%%%%%%%%%%%%%%%%%%%%%%%%%%%%%%%%%%%%%%%%%%%%%%

\chapter{Metodología utilizada}

Para abordar el problema decidí separarlo en varias partes:

\begin{itemize}
    \item \textbf{Análisis de necesidades:} Esta fase se centró en comprender las necesidades y objetivos internos del equipo en relación con un laboratorio de amenazas, esencial para orientar el desarrollo del proyecto.
    
    \item \textbf{Investigación y diseño del laboratorio de amenazas:} Se llevó a cabo una investigación exhaustiva de herramientas, recursos y metodologías para el diseño del laboratorio, basándose en las necesidades identificadas en la fase de análisis.
    
    \item \textbf{Implementación:} En esta etapa, se implementaron las herramientas necesarias para establecer la infraestructura básica del laboratorio. Además, se desarrollaron recursos internos que constituyen la metodología actual del laboratorio.
    
    \item \textbf{Documentación y buenas prácticas:} Se creó documentación técnica para la gestión de la infraestructura del laboratorio, junto con documentación que aborda buenas prácticas y recomendaciones internas.
    
    \item \textbf{Despliegue y puesta en uso:} Con la infraestructura en su lugar, la metodología definida, la documentación técnica y las recomendaciones de buenas prácticas, se llevó a cabo una validación práctica mediante pruebas en casos reales.
\end{itemize}

\section{Análisis de necesidades}

Para comprender las necesidades internas del equipo, desarrolle las siguientes actividades:

\begin{itemize}
    \item Identificar las actividades que ya se realizaban cotidianamente como parte del área de seguridad digital, y que podían formar parte del laboratorio de amenazas.
    \item Evaluar las fortalezas y debilidades de estas actividades para determinar qué aspectos mantener y cuáles mejorar.
    \item Definir los objetivos, principios y lineamientos del laboratorio de amenazas, asegurando su alineación con la misión y visión de la organización y del área de seguridad digital.
\end{itemize}

Estas actividades las lleve a cabo mediante reuniones y entrevistas con el equipo interno y la dirección ejecutiva. Identifique tres ejes principales para el laboratorio de amenazas:

\begin{itemize}
    \item \textbf{Análisis y atención de casos:} Ya se realizaban actividades relacionadas con la respuesta rápida ante ataques digitales. Se reconoció la necesidad de estandarizar las prácticas y metodologías de atención a incidentes y análisis técnico, lo que implicaba la creación de manuales operativos y documentos de metodologías estándar internos.
    \item \textbf{Registro de casos:} A pesar de la respuesta rápida existente, la documentación de casos necesitaba mejoras urgentes, especialmente en documentación narrativa, registros cuantitativos, y documentación de indicadores técnicos. Identificamos la necesidad de estandarizar la documentación interna, incluyendo Indicadores de Compromiso (\nomenclature{Indicators of Compromise (IoCs)}{Pendiente.}IoCs) y Tácticas, Técnicas y Procedimientos (\nomenclature{Tactics, Techniques, and Procedures (TTPs)}{Pendiente.}TTPs).
    \item \textbf{Compartir información:} Siguiendo uno de los valores principales de la organización, el conocimiento abierto, decidimos que el trabajo realizado se compartiría con otras organizaciones aliadas de sociedad civil que realizan trabajo similar, facilitando la profesionalización de estas prácticas, que puedan ser revisadas entre pares para su verificación, y que permitan elevar las capacidades técnicas así como las buenas practicas en el sector tecnologico y de seguridad digital en sociedad civil.
\end{itemize}

Es así que de lo anterior establecí los criterios que debía cumplir el diseño e implementación del laboratorio.

\begin{itemize}
    \item Desarrollar y adoptar documentación técnica para la gestión de incidentes, estableciéndola como referencia metodológica técnica.
    \item Implementar capacidades para la documentación y seguimiento de incidentes, enfocándose tanto en el contexto narrativo como en el técnico.
    \item Construir el laboratorio de tal manera que sirva tanto para uso interno de la organización como para la colaboración con otras entidades similares en la región.
    \item Utilizar herramientas basadas en software libre y de código abierto.
\end{itemize}

\section{Investigación y diseño del laboratorio de amenazas}

Realicé un análisis de las herramientas disponibles en el mercado que se alineaban con nuestros criterios de desarrollo. Esta investigación me permitió seleccionar las más adecuadas para nuestras necesidades, diseñando un marco de integración y operación que incluyó elementos complementarios para una funcionalidad completa.

\subsection{Sistematización y mejora de procesos de atención de casos y análisis técnicos}

A pesar de la existencia de prácticas y metodologías previas para la atención de casos y sus análisis técnicos, existía una falta de sistematización que ya se ha mencionado previamente. Mi enfoque se centró en identificar y documentar estas prácticas para establecer un marco de trabajo claro y estandarizado. Entre los procesos documentados se encuentran:

\begin{itemize}
    \item Gestión y respuesta a incidentes, registro de incidentes y casos, y registro de indicadores de compromiso.
    \item Procedimientos de respuesta rápida y triage para diversas amenazas digitales.
    \item Estrategias de inteligencia de amenazas, análisis de riesgo y estudios de infraestructuras maliciosas.
    \item Técnicas de análisis forense adaptadas a diferentes sistemas operativos y dispositivos.
    \item Metodologías de análisis de tráfico de red.
    \item Prácticas de ingeniería inversa y análisis de malware.
\end{itemize}

%\begin{itemize}
%    \item Atención de casos
%    \begin{itemize}
%        \item Protocolo de atención de casos
%        \item Registro de incidentes y casos
%        \item Registro de indicadores de compromiso
%    \end{itemize}
%    \item Respuesta rápida y triage
%    \begin{itemize}
%        \item Phishing
%        \item Equipos de computo
%        \item Dispositivos móviles
%        \item Cuentas en línea
%        \item Sitios web
%    \end{itemize}
%    \item Inteligencia de amenazas
%    \begin{itemize}
%        \item Análisis de riesgo
%        \item Análisis de infraestructura maliciosa
%        \item Filtraciones de datos
%    \end{itemize}
%    \item Análisis forense
%    \begin{itemize}
%        \item Windows
%        \item Mac
%        \item Android
%        \item iOS
%    \end{itemize}
%    \item Análisis de tráfico de red
%    \begin{itemize}
%        \item Captura y análisis de tráfico de red
%        \item Captura y análisis flujo de red
%        \item Captura y análisis de tráfico de red con PiRogue
%    \end{itemize}
%    \item Ingeniería inversa
%    \item Análisis de malware
%\end{itemize}

Para la redacción y estandarización de estos contenidos coordine el diseño de lineamientos de redacción interna basados en la metodología \href{https://diataxis.fr/}{Diátaxis}, la cual es una metodología de redacción de documentación técnica utilizada por distintas empresas como Canonical o CloudFlare.

Para almacenar esta información, era necesario utilizar una herramientas que permitiera una redacción con un lenguaje sencillo, mantener un control de versiones para un trabajo colaborativo con seguimiento de cambios, y compartir la información basado en controles acceso bien definidos.

Así coordine la investigación y evaluación de herramientas que cumplieran estos requerimientos. La herramientas elegida fue \href{https://js.wiki/}{Wiki.JS} la cual permite redactar el contenido con el lenguaje \nomenclature{Markdown}{Pendiente.}markdown, conectar el contenido a un repositorio basado \nomenclature{git}{Pendiente.}git, y definir controles basados en \nomenclature{Traffic Light Protocol (TLP)}{Pendiente}\href{https://www.first.org/tlp/}{TLP}.


%%%%%%%%%%%%%%%%%%%%%%%%%%%%%%%%%%%%%%%%%%%%%%%%%%%%%%%%%%%%%%%%%%%%%%%%%%%%%%%%
%   Resultados
%   Descripción y análisis de los resultados de su participación
%   dentro del proyecto o actividades realizadas, así como de las aportaciones
%   que muestren su capacidad y criterio profesional al aplicar los
%   conocimientos adquiridos durante la carrera.
%%%%%%%%%%%%%%%%%%%%%%%%%%%%%%%%%%%%%%%%%%%%%%%%%%%%%%%%%%%%%%%%%%%%%%%%%%%%%%%%

\chapter{Resultados}

[Descripción y análisis de los resultados de su participación dentro del proyecto o actividades realizadas, así como de las aportaciones que muestren su capacidad y criterio profesional al aplicar los conocimientos adquiridos durante la carrera.]

%%%%%%%%%%%%%%%%%%%%%%%%%%%%%%%%%%%%%%%%%%%%%%%%%%%%%%%%%%%%%%%%%%%%%%%%%%%%%%%%
%   Conclusiones
%   Deben reflejar los logros alcanzados conforme a los objetivos planteados.
%%%%%%%%%%%%%%%%%%%%%%%%%%%%%%%%%%%%%%%%%%%%%%%%%%%%%%%%%%%%%%%%%%%%%%%%%%%%%%%%

\chapter{Conclusiones}

[Deben reflejar los logros alcanzados conforme a los objetivos planteados.]

%%%%%%%%%%%%%%%%%%%%%%%%%%%%%%%%%%%%%%%%%%%%%%%%%%%%%%%%%%%%%%%%%%%%%%%%%%%%%%%%
%   Bibliografía
%   Las principales fuentes consultadas y de apoyo para
%   realizar su trabajo (libros, revistas, multimedia, etc.).
%%%%%%%%%%%%%%%%%%%%%%%%%%%%%%%%%%%%%%%%%%%%%%%%%%%%%%%%%%%%%%%%%%%%%%%%%%%%%%%%

\printbibliography[heading=bibintoc]

%%%%%%%%%%%%%%%%%%%%%%%%%%%%%%%%%%%%%%%%%%%%%%%%%%%%%%%%%%%%%%%%%%%%%%%%%%%%%%%%
%   Apéndices/Anexos
%   Debe contener única y exclusivamente aquella información que
%   ayude al lector a comprender un poco más del asunto, pero que harían
%   una interrupción abrupta a la lectura en caso de incorporarse en el mismo
%   texto.
%%%%%%%%%%%%%%%%%%%%%%%%%%%%%%%%%%%%%%%%%%%%%%%%%%%%%%%%%%%%%%%%%%%%%%%%%%%%%%%%

\appendix

%%%%%%%%%%%%%%%%%%%%%%%%%%%%%%%%%%%%%%%%%%%%%%%%%%%%%%%%%%%%%%%%%%%%%%%%%%%%%%%%
%   Apéndice A
%%%%%%%%%%%%%%%%%%%%%%%%%%%%%%%%%%%%%%%%%%%%%%%%%%%%%%%%%%%%%%%%%%%%%%%%%%%%%%%%

\chapter{Questionnaire}

%%%%%%%%%%%%%%%%%%%%%%%%%%%%%%%%%%%%%%%%%%%%%%%%%%%%%%%%%%%%%%%%%%%%%%%%%%%%%%%%
%   Apéndice B
%%%%%%%%%%%%%%%%%%%%%%%%%%%%%%%%%%%%%%%%%%%%%%%%%%%%%%%%%%%%%%%%%%%%%%%%%%%%%%%%

\chapter{Consent Form}

%%%%%%%%%%%%%%%%%%%%%%%%%%%%%%%%%%%%%%%%%%%%%%%%%%%%%%%%%%%%%%%%%%%%%%%%%%%%%%%%
%   Para generar índices (?)
%%%%%%%%%%%%%%%%%%%%%%%%%%%%%%%%%%%%%%%%%%%%%%%%%%%%%%%%%%%%%%%%%%%%%%%%%%%%%%%%

\printindex

%%%%%%%%%%%%%%%%%%%%%%%%%%%%%%%%%%%%%%%%%%%%%%%%%%%%%%%%%%%%%%%%%%%%%%%%%%%%%%%%
%   Notas a píe de página
%%%%%%%%%%%%%%%%%%%%%%%%%%%%%%%%%%%%%%%%%%%%%%%%%%%%%%%%%%%%%%%%%%%%%%%%%%%%%%%%

\theendnotes

%%%%%%%%%%%%%%%%%%%%%%%%%%%%%%%%%%%%%%%%%%%%%%%%%%%%%%%%%%%%%%%%%%%%%%%%%%%%%%%%
%   Material de mano
%%%%%%%%%%%%%%%%%%%%%%%%%%%%%%%%%%%%%%%%%%%%%%%%%%%%%%%%%%%%%%%%%%%%%%%%%%%%%%%%

%% Pocket materials at the VERY END of thesis
\pocketmaterial
\extrachapter{Pocket Material: Map of Case Study Solar Systems} 

\end{document}

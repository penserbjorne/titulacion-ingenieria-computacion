%%%%%%%%%%%%
%% Please rename this main.tex file and the output PDF to
%% [lastname_firstname_graduationyear]
%% before submission.
%%
%% This .tex file is for use with BibLaTeX. Please use
%% main-bibtex.tex instead if you prefer BibTeX.
%%%%%%%%%%%%

\documentclass[12pt]{caltech_thesis}
\usepackage[hyphens]{url}
\usepackage{hyperref}
\usepackage{lipsum}
\usepackage{graphicx}

\usepackage{todonotes}

%% Tentative: newtx for better-looking Times
\usepackage[spanish]{babel}
\usepackage[utf8]{inputenc}
\usepackage[T1]{fontenc}
\usepackage{newtxtext,newtxmath}

%% Added by penserbjorne
\usepackage{csquotes}

% Must use biblatex to produce the Published Contents and Contributions, per-chapter bibliography (if desired), etc.
\usepackage[
    backend=biber,natbib,
    % IMPORTANT: load a style suitable for your discipline
    style=authoryear
]{biblatex}

% Name of your .bib file(s)
\addbibresource{citas.bib}
%\addbibresource{ownpubs.bib}

\begin{document}

%%%%%%%%%%%%%%%%%%%%%%%%%%%%%%%%%%%%%%%%%%%%%%%%%%%%%%%%%%%%%%%%%%%%%%%%%%%%%%%%
%   Portada
%%%%%%%%%%%%%%%%%%%%%%%%%%%%%%%%%%%%%%%%%%%%%%%%%%%%%%%%%%%%%%%%%%%%%%%%%%%%%%%%

% Do remember to remove the square bracket!
\title{Informe de trabajo profesional sobre el \\diseño e implementación de un laboratorio de amenazas digitales para sociedad civil con software libre y de código abierto}
\author{Paul Sebastian Aguilar Enriquez}

\degreeaward{Ingeniero en Computación}                 % Degree to be awarded
\university{Universidad Nacional Autónoma de México}    % Institution name
\address{Ciudad de México}                     % Institution address
\unilogo{img/01-logo-unam.png}                                 % Institution logo
\copyyear{2024}  % Year (of graduation) on diploma
\defenddate{[Exact Date]}          % Date of defense

%%  If you'd like to remove the Caltech logo from your title page, simply remove the "[logo]" text from the maketitle command
\maketitle[logo]
%\maketitle

%%%%%%%%%%%%%%%%%%%%%%%%%%%%%%%%%%%%%%%%%%%%%%%%%%%%%%%%%%%%%%%%%%%%%%%%%%%%%%%%
%   Licenciamiento
%%%%%%%%%%%%%%%%%%%%%%%%%%%%%%%%%%%%%%%%%%%%%%%%%%%%%%%%%%%%%%%%%%%%%%%%%%%%%%%%

%\orcid{[Author ORCID]}

%% IMPORTANT: Select ONE of the rights statement below.
%\rightsstatement{All rights reserved}
%\rightsstatement{All rights reserved except where otherwise noted}
\rightsstatement{Esta obra está bajo una licencia \\ ``Creative Commons Atribución-NoComercial-CompartirIgual 4.0 Internacional``.}



%%%%%%%%%%%%%%%%%%%%%%%%%%%%%%%%%%%%%%%%%%%%%%%%%%%%%%%%%%%%%%%%%%%%%%%%%%%%%%%%
%   Agradecimientos
%%%%%%%%%%%%%%%%%%%%%%%%%%%%%%%%%%%%%%%%%%%%%%%%%%%%%%%%%%%%%%%%%%%%%%%%%%%%%%%%

\begin{acknowledgements} 	 
   [Add acknowledgements here. If you do not wish to add any to your thesis, you may simply add a blank titled Acknowledgements page.]
\end{acknowledgements}

%%%%%%%%%%%%%%%%%%%%%%%%%%%%%%%%%%%%%%%%%%%%%%%%%%%%%%%%%%%%%%%%%%%%%%%%%%%%%%%%
%   Resumen
%%%%%%%%%%%%%%%%%%%%%%%%%%%%%%%%%%%%%%%%%%%%%%%%%%%%%%%%%%%%%%%%%%%%%%%%%%%%%%%%

\begin{abstract}
   [This abstract must provide a succinct and informative condensation of your work. Candidates are welcome to prepare a lengthier abstract for inclusion in the dissertation, and provide a shorter one in the CaltechTHESIS record.]
\end{abstract}

%%%%%%%%%%%%%%%%%%%%%%%%%%%%%%%%%%%%%%%%%%%%%%%%%%%%%%%%%%%%%%%%%%%%%%%%%%%%%%%%
%   Contenido y contribuciones publicadas
%%%%%%%%%%%%%%%%%%%%%%%%%%%%%%%%%%%%%%%%%%%%%%%%%%%%%%%%%%%%%%%%%%%%%%%%%%%%%%%%

%% Uncomment the `iknowhattodo' option to dismiss the instruction in the PDF.
%\begin{publishedcontent}%[iknowwhattodo]
% List your publications and contributions here.
%\nocite{Cahn:etal:2015,Cahn:etal:2016}
%\end{publishedcontent}

%%%%%%%%%%%%%%%%%%%%%%%%%%%%%%%%%%%%%%%%%%%%%%%%%%%%%%%%%%%%%%%%%%%%%%%%%%%%%%%%
%   Índices
%%%%%%%%%%%%%%%%%%%%%%%%%%%%%%%%%%%%%%%%%%%%%%%%%%%%%%%%%%%%%%%%%%%%%%%%%%%%%%%%

\tableofcontents
\listoffigures
\listoftables
\printnomenclature

\mainmatter

%%%%%%%%%%%%%%%%%%%%%%%%%%%%%%%%%%%%%%%%%%%%%%%%%%%%%%%%%%%%%%%%%%%%%%%%%%%%%%%%
%   http://escolar.ingenieria.unam.mx/_adicionales/titulacion/GuiaInformeTrabajo2019.pdf
%
%   El informe de trabajo profesional es un documento que describe las actividades
%   profesionales que durante o al término de sus estudios desarrolla un alumno por
%   al menos un semestre. Deberá redactarse en primera persona del singular y en
%   pasado, y contemplar algunos de los puntos que se mencionan a continuación,
%   con los que a juicio del asesor y del alumno se demuestren sus capacidades y
%   competencias para el ejercicio de la profesión.
%
%   Por temas legales y/o de confidencialidad laboral y/o de propiedad intelectual,
%   en el trabajo escrito y anexos se debe omitir información específica de la
%   empresa, proyectos, productos, marcas comerciales, etc., procurando emplear
%   en su lugar términos generales al referirse a estos tópicos (“una empresa”, “un
%   proyecto”, “el diseño de un producto”, etc.).
%%%%%%%%%%%%%%%%%%%%%%%%%%%%%%%%%%%%%%%%%%%%%%%%%%%%%%%%%%%%%%%%%%%%%%%%%%%%%%%%

%%%%%%%%%%%%%%%%%%%%%%%%%%%%%%%%%%%%%%%%%%%%%%%%%%%%%%%%%%%%%%%%%%%%%%%%%%%%%%%%
%   Introducción y Objetivo
%   Presentar de forma clara, breve y precisa el
%   contenido del informe, que puede corresponder a un proyecto de mejora,
%   propuesta de cambio o actividades profesionales diversas desempeñadas
%   en el puesto que ocupó, y definir el objetivo de la propuesta, mostrando
%   la capacidad para aplicar los conocimientos adquiridos en su formación
%   como ingeniero.
%%%%%%%%%%%%%%%%%%%%%%%%%%%%%%%%%%%%%%%%%%%%%%%%%%%%%%%%%%%%%%%%%%%%%%%%%%%%%%%%

\chapter{Introducción y Objetivo}

%[Presentar de forma clara, breve y precisa el contenido del informe, que puede corresponder a un proyecto de mejora, propuesta de cambio o actividades profesionales diversas desempeñadas en el puesto que ocupó, y definir el objetivo de la propuesta, mostrando la capacidad para aplicar los conocimientos adquiridos en su formación como ingeniero.]

El presente informe de trabajo recopila las actividades que he desarrollado trabajando en una Organización No Gubernamental (ONG) \nomenclature{ONG}{Organización No Gubernamental} de tecnología cómo coordinador del área de Seguridad Digital y Tecnología.

El objetivo de esta ONG es el de habilitar actores de cambio social a través de la tecnología, por lo que trabajan con diversos grupos e individuos como activistas, periodistas y personas defensoras de derechos en México y América Latina.

El trabajo presentado en este documento corresponde al desarrollo de un Laboratorio de Amenazas \nomenclature[Laboratorio de Amenazas]{Pendiente}{Pendiente} con software libre y de código abierto (Free and Open Source Software, FOSS) \nomenclature[FOSS]{Free and Open Source Software}{Pendiente} para esta ONG en su área de Seguridad Digital, con el objetivo de fortalecer la atención de casos ante incidentes \nomenclature{Incidentes}{Pendiente} o ataques digitales \nomenclature{Ataques Digitales}{Pendiente} realizados en contra de los grupos y personas a los que se acompaña o con los que se trabaja.

Durante el desarrollo del Laboratorio de Amenazas, apliqué competencias fundamentales adquiridas en mi formación en Ingeniería en Computación, tales como el análisis y resolución de problemas, diseño de arquitecturas para infraestructuras web, administración de software, servidores y bases de datos, análisis de seguridad informática, y la implementación de protocolos de respuesta ante incidentes digitales, para fortalecer la infraestructura tecnológica de la ONG y garantizar una respuesta efectiva ante los ataques digitales.

%Here's an example of a citation \citep{GMP81}. Here's another \citep{PP98}. These will appear in the big bibliography at the end of the thesis.

%\begin{figure}[hbt!]
%    \centering
%    \includegraphics[width=.3\textwidth]{img/caltech.png}
%    \caption{This is a figure}\label{fig:logo}
%\end{figure}

%\begin{table}[hbt!]
%    \centering
%    \begin{tabular}{ll}
%        \hline
%        Area & Count\\
%        \hline
%        North & 100\\
%        South & 200\\
%        East & 80\\
%        West & 140\\
%        \hline
%    \end{tabular}
%    \caption{This is a table}\label{tab:sample}
%\end{table}

%Here's an endnote.\endnote{Endnotes are notes that you can use to explain text in a document.}

%%%%%%%%%%%%%%%%%%%%%%%%%%%%%%%%%%%%%%%%%%%%%%%%%%%%%%%%%%%%%%%%%%%%%%%%%%%%%%%%
%   Contexto
%   Capitulo añadido por mi para contextualizar el trabajo que se desarrolla
%%%%%%%%%%%%%%%%%%%%%%%%%%%%%%%%%%%%%%%%%%%%%%%%%%%%%%%%%%%%%%%%%%%%%%%%%%%%%%%%

\chapter{Contexto social}

\section{Sobre agresiones y ataques digitales en América Latina}

Históricamente se ha registrado que los diversos grupos de poder como gobiernos, crimen organizado o empresas privadas han realizado agresiones contra periodistas, activistas y personas defensoras de derechos en la región. Estas practicas en la actualidad se encuentran tanto en el espacio físico como en el digital. \citep{AmnestyInternational-Mexico-2023, Articulo19-Reporte-2022, Articulo19-Reporte-2023, CIMAC-Informe-2022, Internews-Reporte-2023}

Para el trabajo desarrollado aquí es importante mencionar que existen distintas formas de ataques o agresiones digitales que se utilizan contra los grupos e individuos mencionados, los cuales van desde agresiones en redes sociales\nomenclature{Agresiones en redes sociales}{Pendiente.}, doxing\nomenclature{Doxing}{Pendiente.}, intentos de acceso no autorizados a cuentas\nomenclature{Acceso no autorizado}{Pendiente}, phishing\nomenclature{Phishing}{Pendiente.}, ataques de denegación de servicios \nomenclature{DoS}{Ataque de Denegación de servicios} a sitios web, intervención de lineas de comunicación\nomenclature{Intervención de lineas de comunicación}{Pendiente.}, o intervención de dispositivos\nomenclature{Intervención de dispositivos}{Pendiente.} a través de spyware\nomenclature{Spyware}{Pendiente.}, por mencionar algunos de los más habituales que se atienden y trabajan en la ONG.

La evolución de los ataques a estos grupos ha generado un ecosistema complejo con una combinación de componentes físicos, digitales y psicoemocionales para quienes se involucran en el tema, ya sea porque viven las agresiones o porque brindan atención a los mismos. En este sentido, el componente tecnologico juega un papel crucial, en donde los ataques digitales generalmente son la antesala de los ataques físicos. Es así que la atención de los ataques digitales se ha vuelto una prioridad como medida de prevención a otro tipo de ataques. [[--> Inv sobre la conexión de ataques digitales con ataques fisicos , A19? Access Now? Amnistia Internacional?]]

En el contexto latinoamericano existen registros de este tipo de practicas, documentación generalmente desarrollada por grupos de sociedad civil o periodistas y medios que trabajan estos temas. Como se ha mencionado, el nivel de sofisticación es bastante amplio y responde directamente al contexto local que se vive en cada país. Por ejemplo, en Venezuela se ha registrado y documentado la situación sobre el control de Internet \citep{VeSinFiltro-Reporte-2023}; en Nicaragua y Bolivia se ha registrado la promulgación oficial y legal por parte del gobierno para la creación de grupos de monitoreo a través de fuentes abiertas\nomenclature{Fuentes abiertas}{Pendiente.} que permitan identificar y perfilar a personas que el gobierno considere disidentes [[--> Inv sobre NetCenters]]; en México y en El Salvador se ha documentado el uso de tecnología altamente sofisticada para espionaje a través de spyware. [[--> Reportes de CL y AmnestyTech]]

Lo anterior son solo algunos de los ejemplos mejor documentados y que tienen un grado medio o alto de sofisticación o coordinación tecnológica, pero a la par de estos en el día a día los ataques menos sofisticados se encuentran por montones, los cuales suelen tener niveles de sofisticación más bajos pero que se terminan desarrollando a través de la tecnología. Algunos ejemplos de estos son amenazas, acoso y hostigamiento a través de redes sociales o herramientas de mensajería instantánea, intentos de phishing por correo electrónico o mensajería instantánea, o ataques de denegación de servicios a sitios web.

\section{Sobre atención ante agresiones y ataques digitales en América Latina}

En el contexto de sociedad civil en América Latina \nomenclature{Sociedad Civil}{Pendiente.} existen pocos recursos técnicos enfocados a temas de seguridad digital. La mayoría de los que existen tienen un enfoque básico y amigable para personas no técnicas que pertenecen a los grupos vulnerables. Para las personas que atienden casos de ataques y agresiones digitales existen dos posibles escenarios. El primero es pertenecer a un grupo o equipo local, entiéndase esto como un grupo que brinda atención en su comunidad inmediata, o pertenecer a una organización especializada que este adscrita a un proyecto más grande como son grupos de investigación en universidades o en empresas privadas, o a organizaciones internacionales.

En el primero de los casos, estos equipos o grupos no suelen contar con los recursos técnicos, económicos o tecnológicos adecuados para brindar atención especializadas, lo cual se vuelve un problema que lleva a que las atenciones brindadas sean básicas o superficiales, o en muchos casos, insuficientes dadas las limitaciones mencionadas. Cabe mencionar que una gran mayoría de personas que brindan asistencia técnica en este escenario no siempre provienen de una formación profesional o académica en el ámbito de la tecnología, la ingeniería o la ciberseguridad \nomenclature{Ciberseguridad}{Pendiente.}, si no que provienen de grupos de sociedad civil que realizan defensa de derechos digitales \nomenclature{Derechos Digitales}{Pendiente.} con una formación profesional orientada al área de las ciencias sociales y humanidades.

En el segundo caso para poder ingresar a estos grupos o espacios se requiere haber desarrollado una serie de habilidades o proyectos previos especializados (desarrollo de estudios de posgrado o una carrera profesional en ciberseguridad) que limitan el acceso a solo un grupo especifico de personas académicas o profesionistas, que generalmente no se encuentran en contacto directo con los grupos locales que sufren los ataques o agresiones, y que brindan atención selectiva solo a casos lo suficientemente sofisticados que justifiquen el uso de sus recursos. A su vez, estos grupos al ser demasiado especializados y selectivos suelen conformarse de pocas personas, lo que implica en que suelen encontrarse rebasados en capacidad de atención de grupos locales, ya que suelen ser buscados de manera global tanto por los grupos locales que viven las agresiones como por los grupos locales que brindan atención ante las agresiones.

En la mayoría de los casos, las colaboraciones entre estos grupos suelen tener cuellos de botella  por las distintas visiones y prioridades que cada grupo puede tener. Si bien, con el tiempo los grupos especializados han desarrollado herramientas de software libre \nomenclature{Software Libre}{Pendiente.} y de código abierto \nomenclature{Código Abierto}{Pendiente.} (FOSS) \nomenclature{Free and Open Source Software (FOSS)}{Pendiente.} para la atención y canalización de casos con un nivel de sofisticación intermedia, estas suelen requerir conocimiento técnico especializado para su uso, el cual los grupos locales no siempre tienen. Y para la atención de casos más sofisticados, los grupos especializados no suelen compartir su conocimiento, técnicas o metodologías internas ya que consideran que si esta información se vuelve pública pueden darle ventaja a los grupos adversarios para evadir sus técnicas de identificación y atención.

%%%%%%%%%%%%%%%%%%%%%%%%%%%%%%%%%%%%%%%%%%%%%%%%%%%%%%%%%%%%%%%%%%%%%%%%%%%%%%%%
%   Antecedentes
%   Enmarcar en el ámbito de la ingeniería los antecedentes
%   del problema, tema, actividades o proyecto en el que se trabajó.
%%%%%%%%%%%%%%%%%%%%%%%%%%%%%%%%%%%%%%%%%%%%%%%%%%%%%%%%%%%%%%%%%%%%%%%%%%%%%%%%

\chapter{Antecedentes}

[[Enmarcar en el ámbito de la ingeniería los antecedentes del problema, tema, actividades o proyecto en el que se trabajó.]]

\section{Antecedentes dentro de la organización}

Dentro de la organización en la que desarrollé las actividades, el área de Seguridad Digital se enfocaba principalmente en los siguientes puntos:

\begin{itemize}
    \item Análisis de riesgo\nomenclature{Análisis de riesgo}{Pendiente.} a los grupos o individuos que se acompaña.
    \item Desarrollo y fortalecimiento de capacidades\nomenclature{Fortalecimiento de capacidades}{Pendiente} mediante acompañamientos\nomenclature{Acompañamiento}{Pendiente} y talleres\nomenclature{Taller}{Pendiente}.
    \item Desarrollo de políticas\nomenclature{Políticas}{Pendiente.} y protocolos\nomenclature{Protocolos}{Pendiente.}.
    \item Desarrollo de materiales base para usuarios no técnicos, como guías o tutoriales.
    \item Atención a incidentes y ataques digitales.
\end{itemize}

La mayoría de estas actividades eran desarrolladas por una o dos personas dependiendo del nivel de sensibilidad de la actividad. Algunas tenían una base conocida, como era una currícula para talleres, o un marco para el análisis de riesgo, pero en general no existía una documentación base para el desarrollo de la mayoría de ellas. En general el conocimiento no se encontraba sistematizado u organizado, por lo que se generaban silos de información entre los individuos, y la implementación de las actividades correspondía a la experiencia de quienes las realizaban sin seguir lineamientos base o estándar.

En cuestiones de tecnología, se brindaban recomendaciones amigables para los talleres o acompañamientos, correspondiendo estas a la currícula existente. Algunos de los temas técnicos que requieren manejo o implementación de tecnología eran:

\begin{itemize}
    \item Manejo de información (almacenamiento y respaldos seguros).
    \item Configuración y gestión de dispositivos.
    \item Prevención y atención ante phishing y malware\nomenclature{Malware}{Pendiente}.
    \item Cifrado de dispositivos, carpetas o archivos.
    \item Comunicaciones seguras.
    \item Navegación segura (cifrado y proxys)\nomenclature{Proxy}{Pendiente}.
    \item Gestión de contraseñas.
    \item Configuraciones de seguridad y privacidad para cuentas en linea.
    \item Configuraciones de seguridad para sitios web.
\end{itemize}

Dependiendo del nivel de conocimiento o manejo de tecnología, estas herramientas y sus recomendaciones asociadas se podían adaptar a los niveles o necesidades de los grupos o individuos con quienes se estuviera trabajando.

Para la atención de casos, dependiendo del mismo, se brindaban recomendaciones adaptadas al caso y su contexto. Sin embargo, no existía un marco general de atención de casos, una clasificación general de los mismos, o una base de herramientas para análisis o diagnósticos técnicos. Esto corresponde a la problemática explicada en el capitulo anterior sobre las limitaciones y alcances de los grupos locales.

A pesar de esto, el nivel y calidad de las actividades realizadas siempre se considero alto por parte de quienes recibían el apoyo de la organización, destacando su labor en la región como una de las pocas organizaciones que desarrollan estas actividades a ese nivel y con incidencia directa en los grupos y personas a quienes acompañan.

\section{Antecedentes externos a la organización}

En el contexto internacional, donde se encuentran los grupos de investigación y atención especializados ha habido una serie de avances importantes en los últimos años, que afortunadamente en algunos casos se han vuelto públicos o han buscado colaboración directa con las organizaciones locales, lo cual ha permitido un fortalecimiento local considerable.

Algunos de los proyectos, herramientas o recursos que sirvieron como base o inspiración para el trabajo desarrollado son los siguientes.

\subsection{Materiales para usuarios no técnicos}

\subsubsection{Security-in-a-Box}

\textit{Secrity-in-a-Box} (https://securityinabox.org/es/about/) es un proyecto desarrollado por Front Line Defenders (https://www.frontlinedefenders.org/) y Tactical Technology Collective (https://www.tacticaltech.org/) [[Arreglar referencias]], el cual busca desarrollar guías y tutoriales en distintos idiomas sobre protección de información, contraseñas, comunicaciones, dispositivos, redes y archivos.

Estos recursos buscan explicar y enseñar como implementar medidas técnicas de los temas mencionados a usuarios no técnicos. Es uno de los hubs de conocimiento\nomenclature{Hub de conocimiento}{Pendiente.} de su tipo más difundidos en las referencias internacionales.

\subsubsection{Privacy Tools y PRISM Break}

\textit{Privacy Tools} (https://www.privacytools.io/) y \textit{PRISM Break} (https://prism-break.org/en/) [[Referencias papu]] son dos proyectos de conocimiento libre\nomenclature{Conocimiento libre}{Pendiente.}, desarrollados a partir de revelaciones como las de Edward Snowden en 2013 [[Referecias]] o a través de proyectos como WikiLeaks (y por su fundador Julian Assange) [referencias] sobre proyectos de vigilancia y espionaje masivo por parte de los gobiernos y estados.

El objetivo de estos proyectos es brindar información, recomendaciones y alternativas de herramientas que permitan evadir estos mecanismos o programas de vigilancia y espionaje masivo [Referencias], generalmente a través de herramientas y tecnologías de software libre y de código abierto.

Al igual que \textit{Security-in-a-Box}, estos dos proyectos son de las principales referencias difundidas de su tipo. Siendo mantenidos por una comunidad de aficionados y especialistas de la seguridad, la privacidad y el anonimato digital.

\subsection{Materiales de autoevaluación, y desarrollo de políticas y protocolos}

\subsubsection{Security Planer}

Originalmente \textit{Security Planner} (https://securityplanner.consumerreports.org/es/) fue un proyecto desarrollado por \textit{Citizen Lab} [[Referencias]] (https://citizenlab.ca/2017/12/citizen-lab-launches-security-planner/) en 2017, y que actualmente se encuentra bajo el desarrollo de \textit{Consumer Reports} (https://www.consumerreports.org/).

El objetivo de \textit{Security Planner} es brindar recomendaciones de seguridad digital basadas en las necesidades e intereses de las personas, esto a través de un cuestionario diagnostico interactivo.

El cuestionario permite seleccionar los tipos de dispositivos que la persona este utilizando, las opciones de amenazas de las que se quiere proteger y un par de opciones de temas de interés en términos de seguridad y privacidad. Con base en esto ofrece como resultado un \textit{plan de acción} con recomendaciones especificas derivadas de las opciones seleccionadas.

\subsubsection{Cybersecurity Assessment Tool (CAT)}

\textit{Cybersecurity Assessment Tool (CAT)} (https://www.fordfoundation.org/work/our-grants/building-institutions-and-networks/cybersecurity-assessment-tool/) es un formulario de evaluación desarrollado por un grupo de especialista de la \textit{Fundación FORD} (https://www.fordfoundation.org) durante 2020. Su objetivo es medir la madurez, resilencia y fortalezas de una organización en términos de ciberseguridad. La herramienta esta diseñada para ser respondida en una sesión o entrevista de aproximadamente 30 minutos.

\textit{CAT} contempla elementos de marcos de trabajo de la industria como son el de \textit{Respuesta a Incidentes de SANS} (https://www.sans.org/reading-room/whitepapers/incident/paper/1791), el de \textit{Cibserseguridad de NIST} (https://www.nist.gov/cyberframework) o el de \textit{Gestión de riesgo de la ISO 27005} (https://www.iso.org/standard/75281.html).

Desafortunadamente si bien la herramienta contempla estos marcos de trabajo como referencias, la interpretación y análisis de los resultados, así como las recomendaciones derivadas de estos y sus implementaciones no se proporcionan en la herramienta y requieren el apoyo de un especialista en ciberseguridad. Aún así, \textit{CAT} es una de las pocas opciones disponibles para realizar evaluaciones de este tipo con enfoque en sociedad civil\nomenclature{Organizaciones de Sociedad Civil (OSC)}{Pendiente.}.

\subsubsection{SAFETAG}

\textit{SAFETAG} (https://safetag.org/) es un proyecto metodológico desarrollado por \textit{INTERNEWS} (https://internews.org/resource/safetag/) desde 2017. El objetivo es proporcionar un marco de trabajo que adapta las metodologías de pruebas de penetración\nomenclature{Pruebas de penetración}{Pendiente.} y de análisis de riesgo de la industria privada al contexto de organizaciones de sociedad civil de pequeña y mediana escala. 

La relevancia de \textit{SAFETAG} es ser el primer marco de trabajo completo de cibserguridad enfocado a sociedad civil (y único hasta el momento), que contempla procesos metodológicos profesionales adaptados a las necesidades y contextos de los grupos que requieren o que brindan el apoyo de atención o acompañamiento para procesos institucionales de seguridad digital.

A diferencia de \textit{CAT} cuyo enfoque es solamente de evaluación, \textit{SAFETAG} si contempla todo un proceso de evaluación, pruebas, recomendaciones, implementación y seguimiento.  En cualquier caso, ambas herramientas pueden ser complementarias.

El nombre de \textit{SAFETAG} viene de \textit{Secure Auditing Framework and Evaluation Template for Advocacy Groups} que en español se traduce como \textit{Marco de Auditoría Segura y Plantilla de Evaluación para Grupos de Defensa}. 

\subsection{Atención de casos}

\subsubsection{Security Without Borders}

En Diciembre de 2016, Claudio \textit{Nex} Guarnieri (https://nex.sx/) presentó (https://archive.org/details/33C3-Hacking\_the\_World) en el \textit{33 Chaos Communication Congress} (https://en.wikipedia.org/wiki/Chaos\_Communication\_Congress) el proyecto \textit{Security Without Borders} (https://securitywithoutborders.org), el cual busca generar una serie de recursos desarrollados por especialistas de ciberseguridad para apoyar a los grupos de sociedad civil en la atención técnica ante ataques digitales.

Esta serie de recursos se enfocan en recomendaciones para la atención de casos que requieran análisis de phishing, forense digital o accesos no autorizados a cuentas en linea. Los recursos se encuentran disponibles en linea en el sitio web del proyecto (https://securitywithoutborders.org/resources.html) y en GitHub (https://github.com/securitywithoutborders) con un formato de guías que pueden ser seguidos por las personas que brindan atención y apoyo.

El proyecto surgió como respuesta al desequilibrio técnico identificado por \textit{Nex} entre los atacantes y los grupos e individuos que reciben los ataques. Y es uno de los pocos proyectos que tienen un enfoque técnico para el fortalecimiento de capacidades de los grupos en la atención de ataques digitales en contexto de sociedad civil.

\subsection{Herramientas forenses}

\subsubsection{PC Quick Forensics}

\textit{PC Quick Forensics} (https://github.com/botherder/pcqf) o \textit{pcqf} es una herramienta desarrollada por Claudio \textit{Nex} Guarnieri (https://nex.sx/) para automatizar la adquisición de información que pudiera servir para el análisis forense en equipos de computo.

La herramientas obtiene información sobre los procesos en ejecución, las aplicaciones que se lanzan al iniciar de la sesión y una copia de la memoria RAM. 

El enfoque de la herramientas es ser de fácil uso por personas sin conocimiento técnico para que estas puedan ejecutarla y compartir información técnica de sus dispositivos.

Se recomienda cuando no se tiene acceso físico al dispositivo, cuando se realiza una asistencia remota, o cuando no es posible desarrollar procesos como la copia de discos o memorias completas.

\subsubsection{Android Quick Forensics}

\textit{Android Quick Forensics} (https://github.com/botherder/androidqf) o \textit{androidqf} es una herramienta (también) desarrollada por Claudio \textit{Nex} Guarnieri (https://nex.sx/) para automatizar la adquisición de información que pudiera servir para el análisis forense en dispositivos Android.

La herramientas obtiene información sobre las aplicaciones instaladas, una copia de las mismas, información de los procesos en ejecución e información del sistema.

El enfoque de la herramienta (al igual que \textit{pcqf}) es ser de fácil uso por personas sin conocimiento técnico para que estas puedan ejecutarla y compartir información técnica de sus dispositivos.

Igualmente se recomienda cuando no se tiene acceso físico al dispositivo, cuando se realiza una asistencia remota, o cuando no es posible desarrollar procesos como la extracción completa de una imagen de sistema.

\subsubsection{Mobile Forensics Toolkit (MVT)}

\textit{Mobile Forensics Toolkit} (https://github.com/mvt-project/mvt) o \textit{MVT} es una herramienta desarrollada por el \textit{Amnesty International Security Lab} (equipo liderado por Claudio \textit{Nex} Guarnieri (https://nex.sx/)) de \textit{Amnesty Tech} (https://www.amnesty.org/en/tech/), la cual permite automatizar el proceso de extracción y análisis de dispositivos móviles potencialmente comprometidos por malware.

El proyecto fue desarrollado en el contexto del \textit{Pegasus project} (https://forbiddenstories.org/about-the-pegasus-project/) junto con la metodología forense (https://www.amnesty.org/en/latest/research/2021/07/forensic-methodology-report-how-to-catch-nso-groups-pegasus/) de \textit{Amnesty Tech}.

Esta herramienta es la única disponible de manea pública para la detección de spyware de alta gama dirigido a grupos y personas de sociedad civil.

\textit{MVT} utiliza una lista de \textit{indicadores de compromiso} (\textit{IoCs})\nomenclature{Indicators of Compromise}{Pendiente} precargada para spyware conocido enfocado a sociedad civil, pero también permite añadir otras listas, por ejemplo para la detección de \textit{stalkerware}\nomenclature{Stalkerware}{Pendiente.}.

\subsection{Análisis de trafico de red malicioso}

\subsubsection{TinyCheck y SpyGuard}

\textit{TinyCheck} (https://github.com/KasperskyLab/TinyCheck) es un proyecto desarrollado por \textit{KasperskyLab} (https://www.kaspersky.com/) que permite capturar y analizar el trafico de red de un dispositivo para identificar actividad sospechosa.

Para funcionar, \textit{TinyCheck} genera un punto de acceso WiFi al cual se conecta el dispositivo a analizar. La herramienta incorpora \textit{procesos heurísticos}\nomenclature{Procesos heurísticos}{Pendiente} y permite el uso de \textit{IoCs} como los de la \textit{MVT}.

Actualmente existe un \textit{fork}\nomenclature{Fork}{Pendiente} llamado \textit{SpyGuard} (https://github.com/SpyGuard/SpyGuard) el cual esta siendo desarrollado por el autor original de \textit{TinyCheck} Félix Aimé (https://twitter.com/felixaime) (quien ya no trabaja en \textit{KasperskyLab}). El funcionamiento de \textit{SpyGuard} hasta este momento es idéntico al de \textit{TinyCheck}.

En ambas herramientas se recomienda utilizar el dispositivo a analizar por lo menos 10 minutos para generar trafico suficiente, así como la interacción con herramientas y funciones del dispositivo para disparar o activar el malware en cuestión.

Es importante considerar la limitación de que el análisis realizado solo es valido para el periodo en el que el dispositivo estuvo conectado a \textit{TinyCheck} o \textit{SpyGuard}.

\subsubsection{PiRogue}

\subsubsection{Emergency VPN}


%\begin{refsection}
%If you'd like to have separate bibliographies at the end of each chapter, put a \verb|refsection| around the material of each chapter, then cite as usual -- e.g.~\citep{GMP81,Ful83}. Then do a \verb|\printbibliography| just before the \verb|refsection| ends. \index{bibliography!by chapter}

%\printbibliography[heading=subbibliography]
%\end{refsection}

%%%%%%%%%%%%%%%%%%%%%%%%%%%%%%%%%%%%%%%%%%%%%%%%%%%%%%%%%%%%%%%%%%%%%%%%%%%%%%%%
%   Definición del problema o contexto de la participación profesional
%   Describir claramente el proyecto, propuesta de cambio, actividades o
%   problemática a resolver en el contexto de la ingeniería, así como sus
%   alcances.
%%%%%%%%%%%%%%%%%%%%%%%%%%%%%%%%%%%%%%%%%%%%%%%%%%%%%%%%%%%%%%%%%%%%%%%%%%%%%%%%

\chapter{Definición del problema o contexto de la participación profesional}

[Describir claramente el proyecto, propuesta de cambio, actividades o problemática a resolver en el contexto de la ingeniería, así como sus alcances.]

Para explicar las áreas de oportunidad y necesidades dentro de Seguridad Digital, es necesario mencionar los ejes de trabajo dentro de esta, los cuales son:

\begin{itemize}
    \item *Investigación*. Se encarga de la experimentación e investigación de los temas que son necesarios para el área. Lo que puede abarcar el trabajo en metodologías, herramientas, tecnologías, fenomenos o eventos de interes para la organización o los actores a los que se acompaña. Todo el conocimiento generado en este eje de trabajo impacta a los demás.
    \item *Fortalecimiento*. En este eje se desarrollan actividades de fortalemiento de capacidades para los actores con los que se busca trabajar, generalmente en formato de talleres, así como materiales complementarios. Las actividades de fotalecimiento se desarrollan en espacios habilitadores en donde las personas tienen un primer contacto con algún tema de interes el cual usualmente es de nivel básico o inicial.
    \item *Acompañamiento*. Son los procesos de transformación y fortalecimiento continuo que se desarrollan con actores a los que se acompaña. Estos se suelen componer de distinas etapas como son diagnosticos, actividades de fortalemiento y transformación, seguimiento y monitoreo. 
    \item *Atención*. En este eje se realizan las actividades de atención ante incidentes o ataques digitales que reportan los actores a quienes se acompaña, y que requieren una atención o apoyo con perspectiva técnica.
    \item *Comunidad*. Aquí se desarrollan actividades y recursos para comunicar y socializar con personas o grupos de interes todo lo aprendido o desarrollado en los demas ejes de trabajo. Como su nombre lo indica, el objetivo principal es generar comunidad en torno a lo trabajado en el área. 
\end{itemize}

***Poner como nota al píe --Z Todos los ejes de trabajo, a excepción del eje de -atención- son compartidos por las demás áreas de la organización.

Es así que debido a un incremento en la demanda de las actividades de todos los ejes de trabajo **nota al pie** se plantearon los siguientes objetivos con el fin de mejorar el área:

\begin{itemize}
    \item Implementar un sitio web dedicado a Seguridad Digital para compartir todos los recursos y materiales desarrollados en el área.
    \item Actualizar la curricula y los materiales utilizados en las actividades de fortalecimiento y acompañamiento.
    \item Desarrollar materiales de aprendizaje complementarios para los grupos e individuos con los que se trabaja. 
    \item Actualizar y fortalecer los procesos de atención a incidentes y ataques desde una perspectiva técnica.
\end{itemize}

*** La nota al pie --z Despues de 2016, año en el que esta ONG junto con otras contrapartes revelaron el uso de software de espionaje contra activistas, periodistas y personas defensoras de derechos en México, el volumen de solicitudes de apoyo para todos los ejes de trabajo se incremento. Habia más personas y organizaciones consultando a la organización en temas de seguridad, solicitando talleres, acompañamientos y atención a casos.

%\publishedas{Cahn:etal:2015}

%[You can have chapters that were published as part of your thesis. The text style of the body should be single column, as it was submitted to the publisher, not formatted as the publisher did.]

%%%%%%%%%%%%%%%%%%%%%%%%%%%%%%%%%%%%%%%%%%%%%%%%%%%%%%%%%%%%%%%%%%%%%%%%%%%%%%%%
%   Metodología utilizada
%   Describir los métodos, técnicas o procedimientos de ingeniería empleados.
%%%%%%%%%%%%%%%%%%%%%%%%%%%%%%%%%%%%%%%%%%%%%%%%%%%%%%%%%%%%%%%%%%%%%%%%%%%%%%%%

\chapter{Metodología utilizada}

[Describir los métodos, técnicas o procedimientos de ingeniería empleados.]

%%%%%%%%%%%%%%%%%%%%%%%%%%%%%%%%%%%%%%%%%%%%%%%%%%%%%%%%%%%%%%%%%%%%%%%%%%%%%%%%
%   Resultados
%   Descripción y análisis de los resultados de su participación
%   dentro del proyecto o actividades realizadas, así como de las aportaciones
%   que muestren su capacidad y criterio profesional al aplicar los
%   conocimientos adquiridos durante la carrera.
%%%%%%%%%%%%%%%%%%%%%%%%%%%%%%%%%%%%%%%%%%%%%%%%%%%%%%%%%%%%%%%%%%%%%%%%%%%%%%%%

\chapter{Resultados}

[Descripción y análisis de los resultados de su participación dentro del proyecto o actividades realizadas, así como de las aportaciones que muestren su capacidad y criterio profesional al aplicar los conocimientos adquiridos durante la carrera.]

%%%%%%%%%%%%%%%%%%%%%%%%%%%%%%%%%%%%%%%%%%%%%%%%%%%%%%%%%%%%%%%%%%%%%%%%%%%%%%%%
%   Conclusiones
%   Deben reflejar los logros alcanzados conforme a los objetivos planteados.
%%%%%%%%%%%%%%%%%%%%%%%%%%%%%%%%%%%%%%%%%%%%%%%%%%%%%%%%%%%%%%%%%%%%%%%%%%%%%%%%

\chapter{Conclusiones}

[Deben reflejar los logros alcanzados conforme a los objetivos planteados.]

%%%%%%%%%%%%%%%%%%%%%%%%%%%%%%%%%%%%%%%%%%%%%%%%%%%%%%%%%%%%%%%%%%%%%%%%%%%%%%%%
%   Bibliografía
%   Las principales fuentes consultadas y de apoyo para
%   realizar su trabajo (libros, revistas, multimedia, etc.).
%%%%%%%%%%%%%%%%%%%%%%%%%%%%%%%%%%%%%%%%%%%%%%%%%%%%%%%%%%%%%%%%%%%%%%%%%%%%%%%%

\printbibliography[heading=bibintoc]

%%%%%%%%%%%%%%%%%%%%%%%%%%%%%%%%%%%%%%%%%%%%%%%%%%%%%%%%%%%%%%%%%%%%%%%%%%%%%%%%
%   Apéndices/Anexos
%   Debe contener única y exclusivamente aquella información que
%   ayude al lector a comprender un poco más del asunto, pero que harían
%   una interrupción abrupta a la lectura en caso de incorporarse en el mismo
%   texto.
%%%%%%%%%%%%%%%%%%%%%%%%%%%%%%%%%%%%%%%%%%%%%%%%%%%%%%%%%%%%%%%%%%%%%%%%%%%%%%%%

\appendix

%%%%%%%%%%%%%%%%%%%%%%%%%%%%%%%%%%%%%%%%%%%%%%%%%%%%%%%%%%%%%%%%%%%%%%%%%%%%%%%%
%   Apéndice A
%%%%%%%%%%%%%%%%%%%%%%%%%%%%%%%%%%%%%%%%%%%%%%%%%%%%%%%%%%%%%%%%%%%%%%%%%%%%%%%%

\chapter{Questionnaire}

%%%%%%%%%%%%%%%%%%%%%%%%%%%%%%%%%%%%%%%%%%%%%%%%%%%%%%%%%%%%%%%%%%%%%%%%%%%%%%%%
%   Apéndice B
%%%%%%%%%%%%%%%%%%%%%%%%%%%%%%%%%%%%%%%%%%%%%%%%%%%%%%%%%%%%%%%%%%%%%%%%%%%%%%%%

\chapter{Consent Form}

%%%%%%%%%%%%%%%%%%%%%%%%%%%%%%%%%%%%%%%%%%%%%%%%%%%%%%%%%%%%%%%%%%%%%%%%%%%%%%%%
%   Para generar índices (?)
%%%%%%%%%%%%%%%%%%%%%%%%%%%%%%%%%%%%%%%%%%%%%%%%%%%%%%%%%%%%%%%%%%%%%%%%%%%%%%%%

\printindex

%%%%%%%%%%%%%%%%%%%%%%%%%%%%%%%%%%%%%%%%%%%%%%%%%%%%%%%%%%%%%%%%%%%%%%%%%%%%%%%%
%   Notas a píe de página
%%%%%%%%%%%%%%%%%%%%%%%%%%%%%%%%%%%%%%%%%%%%%%%%%%%%%%%%%%%%%%%%%%%%%%%%%%%%%%%%

\theendnotes

%%%%%%%%%%%%%%%%%%%%%%%%%%%%%%%%%%%%%%%%%%%%%%%%%%%%%%%%%%%%%%%%%%%%%%%%%%%%%%%%
%   Material de mano
%%%%%%%%%%%%%%%%%%%%%%%%%%%%%%%%%%%%%%%%%%%%%%%%%%%%%%%%%%%%%%%%%%%%%%%%%%%%%%%%

%% Pocket materials at the VERY END of thesis
\pocketmaterial
\extrachapter{Pocket Material: Map of Case Study Solar Systems} 

\end{document}

%%%%%%%%%%%%
%% Please rename this main.tex file and the output PDF to
%% [lastname_firstname_graduationyear]
%% before submission.
%%
%% This .tex file is for use with BibLaTeX. Please use
%% main-bibtex.tex instead if you prefer BibTeX.
%%%%%%%%%%%%

\documentclass[12pt]{caltech_thesis}
\usepackage[hyphens]{url}
\usepackage{lipsum}
\usepackage{graphicx}

\usepackage{todonotes}

%% Tentative: newtx for better-looking Times
\usepackage[spanish]{babel}
\usepackage[utf8]{inputenc}
\usepackage[T1]{fontenc}
\usepackage{newtxtext,newtxmath}

% Must use biblatex to produce the Published Contents and Contributions, per-chapter bibliography (if desired), etc.
\usepackage[
    backend=biber,natbib,
    % IMPORTANT: load a style suitable for your discipline
    style=authoryear
]{biblatex}

% Name of your .bib file(s)
\addbibresource{example.bib}
\addbibresource{ownpubs.bib}

\begin{document}

%%%%%%%%%%%%%%%%%%%%%%%%%%%%%%%%%%%%%%%%%%%%%%%%%%%%%%%%%%%%%%%%%%%%%%%%%%%%%%%%
%   Portada
%%%%%%%%%%%%%%%%%%%%%%%%%%%%%%%%%%%%%%%%%%%%%%%%%%%%%%%%%%%%%%%%%%%%%%%%%%%%%%%%

% Do remember to remove the square bracket!
\title{Informe de Trabajo Profesional}
\author{Paul Sebastian Aguilar Enriquez}

\degreeaward{Ingeniero en Computación}                 % Degree to be awarded
\university{Universidad Nacional Autónoma de México}    % Institution name
\address{Ciudad de México}                     % Institution address
\unilogo{img/01-logo-unam.png}                                 % Institution logo
\copyyear{2022}  % Year (of graduation) on diploma
\defenddate{[Exact Date]}          % Date of defense

%%  If you'd like to remove the Caltech logo from your title page, simply remove the "[logo]" text from the maketitle command
\maketitle[logo]
%\maketitle

%%%%%%%%%%%%%%%%%%%%%%%%%%%%%%%%%%%%%%%%%%%%%%%%%%%%%%%%%%%%%%%%%%%%%%%%%%%%%%%%
%   Licenciamiento
%%%%%%%%%%%%%%%%%%%%%%%%%%%%%%%%%%%%%%%%%%%%%%%%%%%%%%%%%%%%%%%%%%%%%%%%%%%%%%%%

%\orcid{[Author ORCID]}

%% IMPORTANT: Select ONE of the rights statement below.
%\rightsstatement{All rights reserved}
%\rightsstatement{All rights reserved except where otherwise noted}
\rightsstatement{Esta obra está bajo una licencia \\ ``Creative Commons Atribución-NoComercial-CompartirIgual 4.0 Internacional``.}



%%%%%%%%%%%%%%%%%%%%%%%%%%%%%%%%%%%%%%%%%%%%%%%%%%%%%%%%%%%%%%%%%%%%%%%%%%%%%%%%
%   Agradecimientos
%%%%%%%%%%%%%%%%%%%%%%%%%%%%%%%%%%%%%%%%%%%%%%%%%%%%%%%%%%%%%%%%%%%%%%%%%%%%%%%%

\begin{acknowledgements} 	 
   [Add acknowledgements here. If you do not wish to add any to your thesis, you may simply add a blank titled Acknowledgements page.]
\end{acknowledgements}

%%%%%%%%%%%%%%%%%%%%%%%%%%%%%%%%%%%%%%%%%%%%%%%%%%%%%%%%%%%%%%%%%%%%%%%%%%%%%%%%
%   Resumen
%%%%%%%%%%%%%%%%%%%%%%%%%%%%%%%%%%%%%%%%%%%%%%%%%%%%%%%%%%%%%%%%%%%%%%%%%%%%%%%%

\begin{abstract}
   [This abstract must provide a succinct and informative condensation of your work. Candidates are welcome to prepare a lengthier abstract for inclusion in the dissertation, and provide a shorter one in the CaltechTHESIS record.]
\end{abstract}

%%%%%%%%%%%%%%%%%%%%%%%%%%%%%%%%%%%%%%%%%%%%%%%%%%%%%%%%%%%%%%%%%%%%%%%%%%%%%%%%
%   Contenido y contribuciones publicadas
%%%%%%%%%%%%%%%%%%%%%%%%%%%%%%%%%%%%%%%%%%%%%%%%%%%%%%%%%%%%%%%%%%%%%%%%%%%%%%%%

%% Uncomment the `iknowhattodo' option to dismiss the instruction in the PDF.
\begin{publishedcontent}%[iknowwhattodo]
% List your publications and contributions here.
\nocite{Cahn:etal:2015,Cahn:etal:2016}
\end{publishedcontent}

%%%%%%%%%%%%%%%%%%%%%%%%%%%%%%%%%%%%%%%%%%%%%%%%%%%%%%%%%%%%%%%%%%%%%%%%%%%%%%%%
%   Índices
%%%%%%%%%%%%%%%%%%%%%%%%%%%%%%%%%%%%%%%%%%%%%%%%%%%%%%%%%%%%%%%%%%%%%%%%%%%%%%%%

\tableofcontents
\listoffigures
\listoftables
\printnomenclature

\mainmatter

%%%%%%%%%%%%%%%%%%%%%%%%%%%%%%%%%%%%%%%%%%%%%%%%%%%%%%%%%%%%%%%%%%%%%%%%%%%%%%%%
%   http://escolar.ingenieria.unam.mx/_adicionales/titulacion/GuiaInformeTrabajo2019.pdf
%
%   El informe de trabajo profesional es un documento que describe las actividades
%   profesionales que durante o al término de sus estudios desarrolla un alumno por
%   al menos un semestre. Deberá redactarse en primera persona del singular y en
%   pasado, y contemplar algunos de los puntos que se mencionan a continuación,
%   con los que a juicio del asesor y del alumno se demuestren sus capacidades y
%   competencias para el ejercicio de la profesión.
%
%  Por temas legales y/o de confidencialidad laboral y/o de propiedad intelectual,
%  en el trabajo escrito y anexos se debe omitir información específica de la
%  empresa, proyectos, productos, marcas comerciales, etc., procurando emplear
%  en su lugar términos generales al referirse a estos tópicos (“una empresa”, “un
%  proyecto”, “el diseño de un producto”, etc.).
%%%%%%%%%%%%%%%%%%%%%%%%%%%%%%%%%%%%%%%%%%%%%%%%%%%%%%%%%%%%%%%%%%%%%%%%%%%%%%%%

%%%%%%%%%%%%%%%%%%%%%%%%%%%%%%%%%%%%%%%%%%%%%%%%%%%%%%%%%%%%%%%%%%%%%%%%%%%%%%%%
%   Introducción y Objetivo
%   Presentar de forma clara, breve y precisa el
%   contenido del informe, que puede corresponder a un proyecto de mejora,
%   propuesta de cambio o actividades profesionales diversas desempeñadas
%   en el puesto que ocupó, y definir el objetivo de la propuesta, mostrando
%   la capacidad para aplicar los conocimientos adquiridos en su formación
%   como ingeniero.
%%%%%%%%%%%%%%%%%%%%%%%%%%%%%%%%%%%%%%%%%%%%%%%%%%%%%%%%%%%%%%%%%%%%%%%%%%%%%%%%

\chapter{Introducción y Objetivo}

[Presentar de forma clara, breve y precisa el contenido del informe, que puede corresponder a un proyecto de mejora, propuesta de cambio o actividades profesionales diversas desempeñadas en el puesto que ocupó, y definir el objetivo de la propuesta, mostrando la capacidad para aplicar los conocimientos adquiridos en su formación como ingeniero.]

El presente informe de trabajo recopila las actividades que he desarrollado durante mi estancia laboral en una ONG \nomenclature{ONG}{Organización No Gubernamental} de tecnología. El objetivo de esta ONG es el de habilitar actores de cambio social a través de la tecnología, por lo que trabajan con diversos grupos e individuos como son activistas, periodistas y personas defensoras de derechos humanos

Dentro de esta ONG existen distintas áreas, las actividades que desarrolle pertenecen al área de Seguridad Digital\nomenclature{Seguridad Digital}{Pendiente}. La propuesta de trabajo presentada en este documento se enfoca en las actividades realizadas para el desarrollo de recursos, materiales y metodologías que permitan fortalecer dentro del área la atención de casos ante incidentes \nomenclature{Incidentes}{Pendiente} o ataques digitales \nomenclature{Ataques Digitales}{Pendiente} realizados en contra de los grupos que esta ONG acompaña en América Latina.

%Start off all chapters with \verb|chapter|. \index{chapter!numbered} \verb|\extrachapter| will give you an unnumbered chapter that's added to the Table of Contents. \index{chapter!unnumbered}

%Here's an example of a citation \citep{GMP81}. Here's another \citep{PP98}. These will appear in the big bibliography at the end of the thesis.

%\index{bibliography}

%You can define nomenclatures \index{nomenclature} as you talk about key terms in your thesis. So what's a galaxy? \nomenclature{Galaxy}{A system of stars independent from all other systems}

%\section{This is a Section}
%\lipsum[1-2]

%\begin{figure}[hbt!]
%    \centering
%    \includegraphics[width=.3\textwidth]{img/caltech.png}
%    \caption{This is a figure}\label{fig:logo}
%    \index{figures}
%\end{figure}

%\subsection{This is a subsection}

%\begin{table}[hbt!]
%    \centering
%    \begin{tabular}{ll}
%        \hline
%        Area & Count\\
%        \hline
%        North & 100\\
%        South & 200\\
%        East & 80\\
%        West & 140\\
%        \hline
%    \end{tabular}
%    \caption{This is a table}\label{tab:sample}
%    \index{tables}
%\end{table}

%\lipsum[3] \nomenclature{Asteroid}{A very small planet ranging from 1,000 km to less than one km in diameter. Asteroids are found commonly around other larger planets}

%\lipsum[4-5] 

%Here's an endnote.\endnote{Endnotes are notes that you can use to explain text in a document.}

%\section{This is Another Section}
%\lipsum[6-7] 

%%%%%%%%%%%%%%%%%%%%%%%%%%%%%%%%%%%%%%%%%%%%%%%%%%%%%%%%%%%%%%%%%%%%%%%%%%%%%%%%
%   Contexto
%   Capitulo añadido por mi para contextualizar el trabajo que se desarrolla
%%%%%%%%%%%%%%%%%%%%%%%%%%%%%%%%%%%%%%%%%%%%%%%%%%%%%%%%%%%%%%%%%%%%%%%%%%%%%%%%

\chapter{Contexto social}

\section{Sobre agresiones y ataques digitales en la en América Latina}

Históricamente se ha registrado que los diversos grupos de poder como gobiernos, crimen organizado o empresas privadas han realizado agresiones contra periodistas, activistas y personas defensoras de derechos humanos. Si bien estas practicas generalmente se desarrollaban en el espacio físico, y en algún momento a través de las telecomunicaciones convencionales como la telefonía, en la actualidad estos ataques y agresiones han evolucionado al igual que toda la tecnología en nuestro mundo. [[--> Inv sobre estos registros (Tesis de Mariel?)]]

En el contexto actual existen distintas formas de ataques o agresiones digitales que se utilizan contra los grupos e individuos mencionados, los cuales van desde agresiones en redes sociales \nomenclature{Agresiones en redes sociales}{Pendiente.}, doxxing\nomenclature{Doxxing}{Pendiente.}, intentos de acceso no autorizados a cuentas\nomenclature{Acceso no autorizado}{Pendiente}, phishing\nomenclature{Phishing}{Pendiente.}, intervención de lineas de comunicación\nomenclature{Intervención de lineas de comunicación}{Pendiente.}, o intervención de dispositivos\nomenclature{Intervención de dispositivos}{Pendiente.} a través de spyware\nomenclature{Spyware}{Pendiente.}, por mencionar algunos de los más comunes. [[--> Inv sobre estos ataques de manera global]]

La evolución de los ataques a estos grupos ha generado un ecosistema complejo con una combinación de componentes físicos, digitales y psicoemocionales para quienes se involucran en el tema, ya sea porque viven las agresiones o porque ayudan a la atención de los mismos. En este sentido, el componente tecnologico juega un papel crucial, en donde los ataques digitales generalmente anuncian o son la antesala de los ataques físicos. Es así que la atención de los ataques digitales se ha vuelto una prioridad como medida de prevención a otro tipo de ataques. [[--> Inv sobre la conexión de ataques digitales con ataques fisicos , A19? Access Now? Amnistia Internacional?]]

En el contexto latinoamericano existen registros de este tipo de practicas, documentación generalmente desarrollada por la misma sociedad civil. Como se ha mencionado, el nivel de sofisticación es bastante amplio y responde directamente al contexto local que se vive en cada país. Por ejemplo, en países como Venezuela y Cuba se ha registrado y documentado la situación al rededor del control de Internet [[--> Reportes de VE Sin Filtro]]; en Nicaragua y Bolivia se ha registrado la promulgación oficial y legal por parte del gobierno para la creación de grupos de monitoreo a través de fuentes abiertas\nomenclature{Fuentes abiertas}{Pendiente.} que permitan identificar y perfilar a personas que el gobierno considere disidentes [[--> Inv]]; en México y en El Salvador se ha documentado el uso de tecnología altamente sofisticada para espionaje a través de spyware. [[--> Reportes de CL y AmnestyTech]]

Lo anterior son solo algunos de los ejemplos mejor documentados y que tienen implicaciones tecnológicas sofisticadas, pero a la par de estos en el día a día los ataques menos sofisticados se encuentran por montones [[--> Reportes de A19 y AccessNow]], los cuales suelen tener menos implicaciones tecnológicas pero que se terminan desarrollando a través de la tecnología. Algunos ejemplos de estos son amenazas, acoso y hostigamiento a través de redes sociales o herramientas de mensajería instantánea, intentos de phishing por correo electrónico o mensajería instantánea, o ataques de denegación de servicios a sitios web. [[--> Reportes de A19 y AccessNow]]

\section{Sobre atención ante agresiones y ataques digitales en América Latina}

En el contexto de sociedad civil en América Latina \nomenclature{Sociedad Civil}{Pendiente.} existen pocos recursos técnicos enfocados a temas de seguridad digital. La mayoría de los que existen tienen un enfoque amigable y básico para personas no técnicas que pertenecen a estos grupos vulnerables. Para las personas que atienden casos de ataques y agresiones digitales existen dos posibles escenarios. El primero es pertenecer a un grupo o equipo local, entiéndase esto como un grupo que brinda atención en su comunidad inmediata, o pertenecer a una organización especializada que este adscrita a un proyecto más grande como son grupos de investigación en universidades o en empresas privadas, o a organizaciones internacionales, que generalmente operan fuera de los países latinoamericanos.

En el primero de los casos, estos equipos o grupos no suelen contar con los recursos técnicos, económicos o tecnológicos adecuados para brindar atención especializadas, lo cual se vuelve una problemática compleja cuando las agresiones o ataques son sofisticados y obliga a que las atenciones brindadas sean básicas o superficiales, o en muchos casos, insuficientes dadas las limitaciones de estos grupos o equipos. Cabe mencionar que una gran mayoría de personas que brindan asistencia técnica en este escenario no siempre provienen de una formación profesional o académica en el ámbito de la tecnología, la ingeniería o la ciberseguridad \nomenclature{Ciberseguridad}{Pendiente.}, si no que provienen de grupos de sociedad civil que realizan defensa de derechos digitales \nomenclature{Derechos Digitales}{Pendiente.} o algún tipo de hacktivismo \nomenclature{Hacktivismo}{Pendiente.}.

En el segundo caso para poder ingresar a estos grupos o espacios se requiere haber desarrollado una serie de habilidades o proyectos previos especializados que limitan el acceso a solo un grupo especifico de personas académicas o profesionistas, que generalmente no se encuentran en contacto directo con los grupos locales que sufren los ataques o agresiones, y que brindan atención selectiva solo a casos lo suficientemente sofisticados que justifiquen el uso de sus recursos. A su vez, estos grupos al ser demasiado especializados y selectivos se encuentran rebasados en capacidad de atención a las necesidades locales, ya que suelen ser buscados de manera global tanto por los grupos locales que viven las agresiones como por los grupos locales que brindan atención ante las agresiones.

En la mayoría de los casos, estos grupos suelen colaborar muy poco y a través de procesos que generan demasiados cuellos de botella por la complejidad de contextos en que se desarrollan los ataques o agresiones, así como por las distintas visiones y prioridades que cada grupo suele tener. Si bien, con el tiempo los grupos especializados han desarrollado herramientas de software libre \nomenclature{Software Libre}{Pendiente.} y de código abierto \nomenclature{Código Abierto}{Pendiente.} (FOSS) \nomenclature{Free and Open Source Software (FOSS)}{Pendiente.} para la atención y canalización de casos con un nivel de sofisticación intermedia, estas suelen requerir conocimiento técnico especializado para su uso, el cual los grupos locales no siempre tienen. Y para la atención de casos más sofisticados, los grupos especializados no suelen compartir su conocimiento, técnicas o metodologías internas ya que consideran que si esta información se vuelve pública pueden darle ventaja a los grupos adversarios para evadir sus técnicas de identificación y atención.

%%%%%%%%%%%%%%%%%%%%%%%%%%%%%%%%%%%%%%%%%%%%%%%%%%%%%%%%%%%%%%%%%%%%%%%%%%%%%%%%
%   Antecedentes
%   Enmarcar en el ámbito de la ingeniería los antecedentes
%   del problema, tema, actividades o proyecto en el que se trabajó.
%%%%%%%%%%%%%%%%%%%%%%%%%%%%%%%%%%%%%%%%%%%%%%%%%%%%%%%%%%%%%%%%%%%%%%%%%%%%%%%%

\chapter{Antecedentes}

[[Enmarcar en el ámbito de la ingeniería los antecedentes del problema, tema, actividades o proyecto en el que se trabajó.]]

\section{Antecedentes internos}

Dentro de la organización en la que desarrollé las actividades, el área de Seguridad Digital se enfocaba principalmente en los siguientes puntos:

\begin{itemize}
    \item Análisis de riesgo\nomenclature{Análisis de riesgo}{Pendiente.} a los grupos o individuos que se acompaña
    \item Desarrollo y fortalecimiento de capacidades\nomenclature{Fortalecimiento de capacidades}{Pendiente} mediante acompañamientos\nomenclature{Acompañamiento}{Pendiente} y talleres
    \item Desarrollo de políticas\nomenclature{Políticas}{Pendiente.} y protocolos\nomenclature{Protocolos}{Pendiente.}
    \item Desarrollo de materiales base para usuarios no técnicos, como guías o tutoriales
    \item Atención a incidentes y ataques digitales
\end{itemize}

La mayoría de estas actividades eran desarrolladas por una o dos personas dependiendo del nivel de sensibilidad de la actividad. Algunas tenían una base conocida, como era una currícula para talleres, o un marco para el análisis de riesgo, pero en general no existía una documentación base para el desarrollo de la mayoría de ellas. En general el conocimiento no se encontraba sistematizado u organizado, por lo que se generaban silos de información entre los individuos, y la implementación de las actividades correspondía a la experiencia de quienes las realizaban sin seguir lineamientos base o estándar.

En cuestiones de tecnología, se brindaban recomendaciones amigables para los talleres o acompañamientos, correspondiendo estas a la currícula existente. Algunos de los temas técnicos que requieren manejo o implementación de tecnología eran:

\begin{itemize}
    \item Manejo de información (almacenamiento y respaldos seguros)
    \item Configuración y gestión de dispositivos
    \item Prevención y atención ante phishing y malware\nomenclature{Malware}{Pendiente}
    \item Cifrado de dispositivos, carpetas o archivos
    \item Comunicaciones seguras
    \item Navegación segura (cifrado y proxys)
    \item Gestión de contraseñas
    \item Configuraciones de seguridad y privacidad para cuentas en linea
    \item Configuraciones de seguridad para sitios web
\end{itemize}

Dependiendo del nivel de conocimiento o manejo de tecnología, estas herramientas y sus recomendaciones asociadas se podían adaptar a los niveles o necesidades de los grupos o individuos con quienes se estuviera trabajando.

Para la atención de casos, dependiendo del mismo, se brindaban recomendaciones adaptadas al caso y su contexto. Sin embargo, no existía un marco general de atención de casos, una clasificación general de los mismos, o una base de herramientas para análisis o diagnósticos técnicos. Esto corresponde a la problemática explicada en el capitulo anterior sobre las limitaciones y alcances de los grupos locales.

A pesar de esto, el nivel y calidad de las actividades realizadas siempre se considero alto por parte de quienes recibían el apoyo de la organización, destacando su labor en la región como una de las pocas organizaciones que desarrollan estas actividades a ese nivel y con incidencia directa en los grupos y personas a quienes acompañan.

\section{Antecedentes externos}

En el contexto internacional, donde se encuentran los grupos de investigación y atención especializados ha habido una serie de avances importantes en los últimos años, que afortunadamente en algunos casos se han vuelto públicos o han buscado colaboración directa con las organizaciones locales.

Algunos de los proyectos, herramientas o recursos que sirvieron como base o inspiración para el trabajo desarrollado son los siguientes.

\subsection{Security-In-A-Box}

\lipsum[1]

\subsection{PrivaciTools.io}

\lipsum[1]

\subsection{Security Planer}

\lipsum[1]

\subsection{CAT}

\lipsum[1]


\subsection{Safetag}

\lipsum[1]

\subsection{Security Without Borders}

\lipsum[1]

\subsection{Amnesty Tech, tools}

\lipsum[1]

\subsection{TinyCheck}

\lipsum[1]

\subsection{StratoSphere - CivilSphere, tools}

\lipsum[1]

\subsection{Otras herramientas}

\subsubsection{Have I been PWNED}

\subsubsection{Firefox Monitor}

\subsubsection{How secure is my password}

%\begin{refsection}
%If you'd like to have separate bibliographies at the end of each chapter, put a \verb|refsection| around the material of each chapter, then cite as usual -- e.g.~\citep{GMP81,Ful83}. Then do a \verb|\printbibliography| just before the \verb|refsection| ends. \index{bibliography!by chapter}

%\printbibliography[heading=subbibliography]
%\end{refsection}

%%%%%%%%%%%%%%%%%%%%%%%%%%%%%%%%%%%%%%%%%%%%%%%%%%%%%%%%%%%%%%%%%%%%%%%%%%%%%%%%
%   Definición del problema o contexto de la participación profesional
%   Describir claramente el proyecto, propuesta de cambio, actividades o
%   problemática a resolver en el contexto de la ingeniería, así como sus
%   alcances.
%%%%%%%%%%%%%%%%%%%%%%%%%%%%%%%%%%%%%%%%%%%%%%%%%%%%%%%%%%%%%%%%%%%%%%%%%%%%%%%%

\chapter{Definición del problema o contexto de la participación profesional}

[Describir claramente el proyecto, propuesta de cambio, actividades o problemática a resolver en el contexto de la ingeniería, así como sus alcances.]

%\publishedas{Cahn:etal:2015}

%[You can have chapters that were published as part of your thesis. The text style of the body should be single column, as it was submitted to the publisher, not formatted as the publisher did.]

%%%%%%%%%%%%%%%%%%%%%%%%%%%%%%%%%%%%%%%%%%%%%%%%%%%%%%%%%%%%%%%%%%%%%%%%%%%%%%%%
%   Metodología utilizada
%   Describir los métodos, técnicas o procedimientos de ingeniería empleados.
%%%%%%%%%%%%%%%%%%%%%%%%%%%%%%%%%%%%%%%%%%%%%%%%%%%%%%%%%%%%%%%%%%%%%%%%%%%%%%%%

\chapter{Metodología utilizada}

[Describir los métodos, técnicas o procedimientos de ingeniería empleados.]

%%%%%%%%%%%%%%%%%%%%%%%%%%%%%%%%%%%%%%%%%%%%%%%%%%%%%%%%%%%%%%%%%%%%%%%%%%%%%%%%
%   Resultados
%   Descripción y análisis de los resultados de su participación
%   dentro del proyecto o actividades realizadas, así como de las aportaciones
%   que muestren su capacidad y criterio profesional al aplicar los
%   conocimientos adquiridos durante la carrera.
%%%%%%%%%%%%%%%%%%%%%%%%%%%%%%%%%%%%%%%%%%%%%%%%%%%%%%%%%%%%%%%%%%%%%%%%%%%%%%%%

\chapter{Resultados}

[Descripción y análisis de los resultados de su participación dentro del proyecto o actividades realizadas, así como de las aportaciones que muestren su capacidad y criterio profesional al aplicar los conocimientos adquiridos durante la carrera.]

%%%%%%%%%%%%%%%%%%%%%%%%%%%%%%%%%%%%%%%%%%%%%%%%%%%%%%%%%%%%%%%%%%%%%%%%%%%%%%%%
%   Conclusiones
%   Deben reflejar los logros alcanzados conforme a los objetivos planteados.
%%%%%%%%%%%%%%%%%%%%%%%%%%%%%%%%%%%%%%%%%%%%%%%%%%%%%%%%%%%%%%%%%%%%%%%%%%%%%%%%

\chapter{Conclusiones}

[Deben reflejar los logros alcanzados conforme a los objetivos planteados.]

%%%%%%%%%%%%%%%%%%%%%%%%%%%%%%%%%%%%%%%%%%%%%%%%%%%%%%%%%%%%%%%%%%%%%%%%%%%%%%%%
%   Bibliografía
%   Las principales fuentes consultadas y de apoyo para
%   realizar su trabajo (libros, revistas, multimedia, etc.).
%%%%%%%%%%%%%%%%%%%%%%%%%%%%%%%%%%%%%%%%%%%%%%%%%%%%%%%%%%%%%%%%%%%%%%%%%%%%%%%%

\printbibliography[heading=bibintoc]

%%%%%%%%%%%%%%%%%%%%%%%%%%%%%%%%%%%%%%%%%%%%%%%%%%%%%%%%%%%%%%%%%%%%%%%%%%%%%%%%
%   Apéndices/Anexos
%   Debe contener única y exclusivamente aquella información que
%   ayude al lector a comprender un poco más del asunto, pero que harían
%   una interrupción abrupta a la lectura en caso de incorporarse en el mismo
%   texto.
%%%%%%%%%%%%%%%%%%%%%%%%%%%%%%%%%%%%%%%%%%%%%%%%%%%%%%%%%%%%%%%%%%%%%%%%%%%%%%%%

\appendix

%%%%%%%%%%%%%%%%%%%%%%%%%%%%%%%%%%%%%%%%%%%%%%%%%%%%%%%%%%%%%%%%%%%%%%%%%%%%%%%%
%   Apéndice A
%%%%%%%%%%%%%%%%%%%%%%%%%%%%%%%%%%%%%%%%%%%%%%%%%%%%%%%%%%%%%%%%%%%%%%%%%%%%%%%%

\chapter{Questionnaire}

%%%%%%%%%%%%%%%%%%%%%%%%%%%%%%%%%%%%%%%%%%%%%%%%%%%%%%%%%%%%%%%%%%%%%%%%%%%%%%%%
%   Apéndice B
%%%%%%%%%%%%%%%%%%%%%%%%%%%%%%%%%%%%%%%%%%%%%%%%%%%%%%%%%%%%%%%%%%%%%%%%%%%%%%%%

\chapter{Consent Form}

%%%%%%%%%%%%%%%%%%%%%%%%%%%%%%%%%%%%%%%%%%%%%%%%%%%%%%%%%%%%%%%%%%%%%%%%%%%%%%%%
%   Para generar índices (?)
%%%%%%%%%%%%%%%%%%%%%%%%%%%%%%%%%%%%%%%%%%%%%%%%%%%%%%%%%%%%%%%%%%%%%%%%%%%%%%%%

\printindex

%%%%%%%%%%%%%%%%%%%%%%%%%%%%%%%%%%%%%%%%%%%%%%%%%%%%%%%%%%%%%%%%%%%%%%%%%%%%%%%%
%   Notas a píe de página
%%%%%%%%%%%%%%%%%%%%%%%%%%%%%%%%%%%%%%%%%%%%%%%%%%%%%%%%%%%%%%%%%%%%%%%%%%%%%%%%

\theendnotes

%%%%%%%%%%%%%%%%%%%%%%%%%%%%%%%%%%%%%%%%%%%%%%%%%%%%%%%%%%%%%%%%%%%%%%%%%%%%%%%%
%   Material de mano
%%%%%%%%%%%%%%%%%%%%%%%%%%%%%%%%%%%%%%%%%%%%%%%%%%%%%%%%%%%%%%%%%%%%%%%%%%%%%%%%

%% Pocket materials at the VERY END of thesis
\pocketmaterial
\extrachapter{Pocket Material: Map of Case Study Solar Systems} 

\end{document}